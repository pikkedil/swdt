\chapter{A historical view of winding design}
In the first of his two remarkable papers \cite{REF-00835, REF-00836} explains the systematics of stator windings and the calculation of the winding factors. The aim in this work was to determine the parameters that characterise the air gap mmf of the winding. Also in this paper the induced voltage in the coil sides is already mentioned and represented as a vector. The resultant vector diagram was called the star of coil groups (German: ``Spulengruppenstern''). The adjacent vectors on such a diagram that belong to the same phase is called a phase belt (German: ``Zone'').
\index{star of coil groups}
\index{Spulengruppenstern}
\index{Zone}

Two years later, in the second paper by \cite{REF-00837}, the algebraic methods developed in the first paper were visualised by means of Tingley's diagram. The latter could be referred to as a linear representation of the now called star of slots (German: ``Nutenstern''). The star of slots is constructed using the electrical angle between two adjacent slots. Computer technology as we know it today was not available at the time and the use of graph paper certainly was common. Furthermore, such graphical methods definitely contributed to the subject of stator windings.
\index{Tingley's diagram}
\index{Nutenstern}

Vil\'em Kl\'ima (Wilhelm Kauders) died on 6 October 1985 and in an obituary by \cite{REF-01054} it is mentioned that Kl\'ima's equation for the distribution factor\footnote{Also known as the breadth factor.} of fractional slot windings is not found in textbooks. Another remark in the obituary is that in some references it is stated that it is not possible to find a closed-form expression for the winding factor of fractional slot windings. In the rest of this literature overview, none of the authors (except Kremser) refers to or makes use of Kl\'ima's closed-form expression. 

More than fifty years later than \cite{REF-00837}, \cite{REF-00266} introduced a comprehensive study of fractional slot windings and the currents in the parallel paths of rotating machines. The reason for this big time gap does not mean that nothing had happened in the mean  time in winding design; it should be kept in mind that this overview only points out some important aspects. In the first part of Kremser's doctoral thesis a detailed algebraic description of single and double layer fractional slot windings is presented. In order to allocate the coils to the stator slots, modular arithmetic, which uses the so-called commutator pitch, is used. This can be seen as a step toward using computer programs to automatically performing winding designs. In the star of slots the vectors start wrapping, and once put on a graph, two vectors having angles of \ang{45} and \ang{405} respectively lie exactly on each other. The star of slots is thus the same as the remainder after dividing an angle by \ang{360} which can easily be implemented by means of the modulo function.  
\index{modular arithmetic} 
\index{commutator pitch}  

Reference \cite{REF-00452} explains another algebraic method to design stator windings which has much in common with that presented by \cite{REF-00266}\footnote{It should however be noticed that the authors Wach and Kremser are from Poland and Germany respectively.}.  The methods presented in the paper is characterised by the matrix representation of a winding. The winding matrix simplify the winding factor calculation and avoids complex equations. Basically the vectors of the star of slots is used in matrix form. What is called the winding matrix could be seen as the matrix representation of Tingley's diagram. A drawback of the method is that each matrix entry for double layer windings has two values. In computer programming this should be solved by either a multi-dimensional matrix or two two-dimensional matrices. In another paper by \cite{REF-00454}, the focus is on the optimisation of fractional slot windings. In spite of the fact that this paper is a very good source on the topic of winding design, none of the literature in the rest of this section refers to it.
 
Reference \cite{REF-00754} showed that the use of stator coils wound around the stator teeth has some attractive advantages for manufacturing. The coils do not overlap which means that the manufacturing is simplified and the winding overhang is less than that of overlapping windings. Electrically the shorter winding overhang means the use of less copper. Furthermore, the non-overlapping windings are derived from single or double layer overlapping windings. It is also pointed out that in the case of single layer non-overlapping windings the coil pitch can be increased and used as a design parameter\footnote{See Fig.~\ref{fig:f_slotstar} for regular and irregular distributed slots.}. The latter, however, is not taken into account in the winding factor calculation. This paper is referenced quite often, proving the interest among machine designers in non-overlapping windings.  

Reference \cite{REF-01055} did an extensive study on fractional slot windings for low speed applications. The main objective in the doctoral thesis is to compare different slot combinations of a machine with a fixed air gap diameter in the \SI{45}{kW} range. Although both single and double non-overlapping windings are taken into account, the focus is on the double layer type. Since round wire coils are used in the investigated application it is appropriate to use a double layer rather than single layer. If form-wound coils were used the choice of double layer windings would have caused some difficulties when inserting the coils into the stator slots. Also, the stator had semi-closed slots. The star of slots is referred to as the voltage vector graph in Salminen's thesis. A very important comment by the author is that care should be taken when the winding factors for different winding types are to be calculated. Depending on the winding type the right set of equations should be chosen. 

Reference \cite{REF-00756} proposed a method to calculate the winding factors for concentrated coils without knowledge of the winding layout. This is of course sufficient to quickly compare different slot and pole combinations. Using this method it is shown that the best number of slots per pole and phase should be in the range \textonequarter $\leq q\leq$ \textonehalf. 

The tutorial presented by \cite{REF-01056} is an in-depth study of fractional slot windings in general. In their tutorial notes a distinction is made between overlapping and non-overlapping windings. Here too the non-overlapping type is derived from a double layer overlapping winding. 