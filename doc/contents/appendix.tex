\begin{appendices}

\chapter{Discrete Fourier Transform}
In this report it is assumed that the harmonic order of the spatial signal is known. The DFT is primarily used to identify the amplitude of the harmonics. Furthermore, the considered function is periodic: $f(x_1)=f(x_{N+1})$.

The base function used in the analysis is the $\cos(\theta-\phi)$ function, where $\phi$ is the phase shift. The visualisation for different values of phase shift is shown in Fig.~\ref{fig:base}.  

\begin{figure}[htbp]
  \centering
  \fontsize{6}{0}\selectfont
  \setlength\fwidth{0.26\textwidth}
  \subfloat[$\cos\bigl(\theta-(-30)\bigr)$]{
  % This file was created by matlab2tikz.
%
%The latest updates can be retrieved from
%  http://www.mathworks.com/matlabcentral/fileexchange/22022-matlab2tikz-matlab2tikz
%where you can also make suggestions and rate matlab2tikz.
%
\begin{tikzpicture}

\begin{axis}[%
width=0.951\fwidth,
height=0.75\fwidth,
at={(0\fwidth,0\fwidth)},
scale only axis,
xmin=-180,
xmax=180,
xtick={-180,    0,  180},
xlabel={$\theta/\SI{}{\arcdeg}$},
xmajorgrids,
ymin=-0.999874127673875,
ymax=0.999874127673875,
ylabel={$\cos(\theta-\phi)$},
ymajorgrids,
axis background/.style={fill=white},
legend style={legend cell align=left,align=left,legend plot pos=left,draw=black}
]
\addplot [color=black,solid]
  table[row sep=crcr]{%
-180	-0.866025403784439\\
-176.363636363636	-0.832569854634771\\
-172.727272727273	-0.795761840530832\\
-169.090909090909	-0.755749574354258\\
-165.454545454545	-0.712694171378863\\
-161.818181818182	-0.666769000516292\\
-158.181818181818	-0.618158986220605\\
-154.545454545455	-0.567059863862771\\
-150.909090909091	-0.513677391573407\\
-147.272727272727	-0.45822652172741\\
-143.636363636364	-0.400930535406614\\
-140	-0.342020143325669\\
-136.363636363636	-0.28173255684143\\
-132.727272727273	-0.220310532786541\\
-129.090909090909	-0.15800139597335\\
-125.454545454545	-0.0950560433041827\\
-121.818181818182	-0.0317279334980679\\
-118.181818181818	0.0317279334980676\\
-114.545454545455	0.0950560433041828\\
-110.909090909091	0.15800139597335\\
-107.272727272727	0.220310532786541\\
-103.636363636364	0.28173255684143\\
-100	0.342020143325669\\
-96.3636363636364	0.400930535406614\\
-92.7272727272727	0.45822652172741\\
-89.0909090909091	0.513677391573407\\
-85.4545454545455	0.567059863862771\\
-81.8181818181818	0.618158986220605\\
-78.1818181818182	0.666769000516292\\
-74.5454545454545	0.712694171378863\\
-70.9090909090909	0.755749574354258\\
-67.2727272727273	0.795761840530832\\
-63.6363636363636	0.832569854634771\\
-60	0.866025403784439\\
-56.3636363636364	0.895993774291336\\
-52.7272727272727	0.922354294104581\\
-49.0909090909091	0.945000818714668\\
-45.4545454545454	0.963842158559942\\
-41.8181818181818	0.978802446214779\\
-38.1818181818182	0.989821441880933\\
-34.5454545454545	0.996854775951942\\
-30.9090909090909	0.999874127673875\\
-27.2727272727273	0.998867339183008\\
-23.6363636363636	0.993838464461254\\
-20	0.984807753012208\\
-16.3636363636364	0.971811568323542\\
-12.7272727272727	0.954902241444074\\
-9.09090909090908	0.934147860265107\\
-5.45454545454543	0.909631995354518\\
-1.81818181818181	0.881453363447582\\
1.81818181818184	0.849725429949514\\
5.45454545454546	0.814575952050336\\
9.09090909090911	0.776146464291757\\
12.7272727272727	0.734591708657533\\
16.3636363636364	0.690079011482112\\
20	0.642787609686539\\
23.6363636363637	0.59290792905464\\
27.2727272727273	0.540640817455598\\
30.9090909090909	0.486196736100469\\
34.5454545454545	0.429794912089172\\
38.1818181818182	0.371662455660328\\
41.8181818181818	0.312033445698487\\
45.4545454545455	0.251147987181079\\
49.0909090909091	0.18925124436041\\
52.7272727272727	0.126592453573749\\
56.3636363636364	0.0634239196565648\\
60	-3.82858892158944e-016\\
63.6363636363637	-0.0634239196565647\\
67.2727272727273	-0.126592453573749\\
70.9090909090909	-0.18925124436041\\
74.5454545454546	-0.25114798718108\\
78.1818181818182	-0.312033445698487\\
81.8181818181818	-0.371662455660327\\
85.4545454545455	-0.429794912089171\\
89.0909090909091	-0.486196736100469\\
92.7272727272727	-0.540640817455598\\
96.3636363636364	-0.59290792905464\\
100	-0.642787609686539\\
103.636363636364	-0.690079011482112\\
107.272727272727	-0.734591708657533\\
110.909090909091	-0.776146464291757\\
114.545454545455	-0.814575952050336\\
118.181818181818	-0.849725429949514\\
121.818181818182	-0.881453363447582\\
125.454545454545	-0.909631995354518\\
129.090909090909	-0.934147860265107\\
132.727272727273	-0.954902241444074\\
136.363636363636	-0.971811568323542\\
140	-0.984807753012208\\
143.636363636364	-0.993838464461254\\
147.272727272727	-0.998867339183008\\
150.909090909091	-0.999874127673875\\
154.545454545455	-0.996854775951942\\
158.181818181818	-0.989821441880933\\
161.818181818182	-0.978802446214779\\
165.454545454545	-0.963842158559942\\
169.090909090909	-0.945000818714668\\
172.727272727273	-0.922354294104581\\
176.363636363636	-0.895993774291336\\
180	-0.866025403784439\\
};

\addplot [color=blue,only marks,mark=o,mark options={solid}]
  table[row sep=crcr]{%
-180	-0.866025403784439\\
-152.307692307692	-0.534465805383544\\
-124.615384615385	-0.080466567836728\\
-96.9230769230769	0.391966606025713\\
-69.2307692307692	0.774604929331893\\
-41.5384615384616	0.979790648616705\\
-13.8461538461539	0.960518081924436\\
13.8461538461538	0.721202425731447\\
41.5384615384615	0.316667992325635\\
69.2307692307692	-0.160411274341826\\
96.9230769230769	-0.600742256776532\\
124.615384615385	-0.903450415706285\\
152.307692307692	-0.999188960281069\\
180	-0.866025403784439\\
};

\addplot [color=red,dashed]
  table[row sep=crcr]{%
-180	-0.866025384341204\\
-176.363636363636	-0.832569835950002\\
-172.727272727273	-0.795761822679717\\
-169.090909090909	-0.755749557408629\\
-165.454545454545	-0.712694155406906\\
-161.818181818182	-0.666768985582272\\
-158.181818181818	-0.61815897238461\\
-154.545454545455	-0.567059851180464\\
-150.909090909091	-0.513677380095808\\
-147.272727272727	-0.458226511500688\\
-143.636363636364	-0.4009305264719\\
-140	-0.342020135718892\\
-136.363636363636	-0.281732550593172\\
-132.727272727273	-0.220310527921913\\
-129.090909090909	-0.158001392511893\\
-125.454545454545	-0.0950560412597869\\
-121.818181818182	-0.0317279328789172\\
-118.181818181818	0.0317279326895279\\
-114.545454545455	0.0950560410712561\\
-110.909090909091	0.158001392325076\\
-107.272727272727	0.220310527737657\\
-103.636363636364	0.281732550412314\\
-100	0.342020135542257\\
-96.3636363636364	0.400930526300295\\
-92.7272727272727	0.458226511334901\\
-89.0909090909091	0.513677379936601\\
-85.4545454545455	0.567059851028574\\
-81.8181818181818	0.618158972240744\\
-78.1818181818182	0.666768985447106\\
-74.5454545454545	0.712694155281079\\
-70.9090909090909	0.755749557292744\\
-67.2727272727273	0.795761822574336\\
-63.6363636363636	0.832569835855645\\
-60	0.866025384258348\\
-56.3636363636364	0.895993754096953\\
-52.7272727272727	0.92235427332327\\
-49.0909090909091	0.945000797430155\\
-45.4545454545454	0.96384213685798\\
-41.8181818181818	0.978802424182802\\
-38.1818181818182	0.989821419607705\\
-34.5454545454545	0.996854753527197\\
-30.9090909090909	0.999874105187957\\
-27.2727272727273	0.998867316726507\\
-23.6363636363636	0.993838442124644\\
-20	0.984807730885477\\
-16.3636363636364	0.971811546495835\\
-12.7272727272727	0.954902220003331\\
-9.09090909090908	0.934147839297711\\
-5.45454545454543	0.909631974944945\\
-1.81818181818181	0.881453343678061\\
1.81818181818184	0.849725410899699\\
5.45454545454546	0.814575933796981\\
9.09090909090911	0.77614644690841\\
12.7272727272727	0.734591692214239\\
16.3636363636364	0.690078996045129\\
20	0.642787595318076\\
23.6363636363637	0.5929079158126\\
27.2727272727273	0.54064080539335\\
30.9090909090909	0.486196725266632\\
34.5454545454545	0.429794902527417\\
38.1818181818182	0.371662447409205\\
41.8181818181818	0.31203343879127\\
45.4545454545455	0.251147981645627\\
49.0909090909091	0.189251240219061\\
52.7272727272727	0.126592450843226\\
56.3636363636364	0.0634239183479103\\
60	1.18531457202731e-010\\
63.6363636363637	-0.0634239181112759\\
67.2727272727273	-0.126592450607878\\
70.9090909090909	-0.189251239985851\\
74.5454545454546	-0.2511479814154\\
78.1818181818182	-0.312033438564855\\
81.8181818181818	-0.371662447187419\\
85.4545454545455	-0.429794902311057\\
89.0909090909091	-0.486196725056474\\
92.7272727272727	-0.540640805190144\\
96.3636363636364	-0.592907915617068\\
100	-0.64278759513091\\
103.636363636364	-0.690078995866988\\
107.272727272727	-0.734591692045743\\
110.909090909091	-0.776146446750142\\
114.545454545455	-0.814575933649483\\
118.181818181818	-0.849725410763469\\
121.818181818182	-0.881453343553552\\
125.454545454545	-0.909631974832562\\
129.090909090909	-0.93414783919781\\
132.727272727273	-0.95490221991622\\
136.363636363636	-0.971811546421768\\
140	-0.984807730824658\\
143.636363636364	-0.99383844207722\\
147.272727272727	-0.998867316692575\\
150.909090909091	-0.999874105167557\\
154.545454545455	-0.996854753520316\\
158.181818181818	-0.989821419614274\\
161.818181818182	-0.9788024242027\\
165.454545454545	-0.96384213689103\\
169.090909090909	-0.945000797476129\\
172.727272727273	-0.922354273381886\\
176.363636363636	-0.89599375416788\\
180	-0.866025384341205\\
};

\node[right, align=left, text=black]
at (axis cs:-135,-0.7) {input: $\phi = \SI{-30}{\arcdeg}$};
\node[right, align=left, text=black]
at (axis cs:-135,-0.9) {result: $\phi = \SI{-30}{\arcdeg}$};
\end{axis}
\end{tikzpicture}%}
  \hfill
  \subfloat[$\cos(\theta)$]{
  % This file was created by matlab2tikz.
%
%The latest updates can be retrieved from
%  http://www.mathworks.com/matlabcentral/fileexchange/22022-matlab2tikz-matlab2tikz
%where you can also make suggestions and rate matlab2tikz.
%
\begin{tikzpicture}

\begin{axis}[%
width=0.951\fwidth,
height=0.75\fwidth,
at={(0\fwidth,0\fwidth)},
scale only axis,
xmin=-180,
xmax=180,
xtick={-180,    0,  180},
xlabel={$\theta/\SI{}{\arcdeg}$},
xmajorgrids,
ymin=-1,
ymax=0.999496542383185,
ylabel={$\cos(\theta-\phi)$},
ymajorgrids,
axis background/.style={fill=white},
legend style={legend cell align=left,align=left,legend plot pos=left,draw=black}
]
\addplot [color=black,solid]
  table[row sep=crcr]{%
-180	-1\\
-176.363636363636	-0.997986676471884\\
-172.727272727273	-0.991954812830795\\
-169.090909090909	-0.981928697262707\\
-165.454545454545	-0.967948701396356\\
-161.818181818182	-0.950071117740945\\
-158.181818181818	-0.928367933016073\\
-154.545454545455	-0.902926538286621\\
-150.909090909091	-0.873849377069785\\
-147.272727272727	-0.841253532831181\\
-143.636363636364	-0.805270257531059\\
-140	-0.766044443118978\\
-136.363636363636	-0.72373403810507\\
-132.727272727273	-0.678509411557132\\
-129.090909090909	-0.630552667084522\\
-125.454545454545	-0.580056909571198\\
-121.818181818182	-0.527225467610502\\
-118.181818181818	-0.472271074772683\\
-114.545454545455	-0.415415013001886\\
-110.909090909091	-0.356886221591872\\
-107.272727272727	-0.296920375328275\\
-103.636363636364	-0.235758935509427\\
-100	-0.17364817766693\\
-96.3636363636364	-0.110838199901011\\
-92.7272727272727	-0.0475819158237421\\
-89.0909090909091	0.0158659638348082\\
-85.4545454545455	0.0792499568567887\\
-81.8181818181818	0.142314838273285\\
-78.1818181818182	0.204806668065191\\
-74.5454545454545	0.266473813690035\\
-70.9090909090909	0.327067963317422\\
-67.2727272727273	0.386345125693129\\
-63.6363636363636	0.444066612605774\\
-60	0.5\\
-56.3636363636364	0.55392006386611\\
-52.7272727272727	0.605609687137667\\
-49.0909090909091	0.654860733945285\\
-45.4545454545454	0.701474887706321\\
-41.8181818181818	0.745264449675755\\
-38.1818181818182	0.786053094742788\\
-34.5454545454545	0.823676581429833\\
-30.9090909090909	0.857983413234977\\
-27.2727272727273	0.888835448654923\\
-23.6363636363636	0.91610845743207\\
-20	0.939692620785908\\
-16.3636363636364	0.959492973614497\\
-12.7272727272727	0.975429786885407\\
-9.09090909090908	0.987438888676394\\
-5.45454545454543	0.995471922573085\\
-1.81818181818181	0.999496542383185\\
1.81818181818184	0.999496542383185\\
5.45454545454546	0.995471922573085\\
9.09090909090911	0.987438888676394\\
12.7272727272727	0.975429786885407\\
16.3636363636364	0.959492973614497\\
20	0.939692620785908\\
23.6363636363637	0.916108457432069\\
27.2727272727273	0.888835448654923\\
30.9090909090909	0.857983413234977\\
34.5454545454545	0.823676581429833\\
38.1818181818182	0.786053094742787\\
41.8181818181818	0.745264449675755\\
45.4545454545455	0.701474887706321\\
49.0909090909091	0.654860733945285\\
52.7272727272727	0.605609687137667\\
56.3636363636364	0.55392006386611\\
60	0.5\\
63.6363636363637	0.444066612605774\\
67.2727272727273	0.386345125693129\\
70.9090909090909	0.327067963317422\\
74.5454545454546	0.266473813690035\\
78.1818181818182	0.20480666806519\\
81.8181818181818	0.142314838273285\\
85.4545454545455	0.0792499568567887\\
89.0909090909091	0.0158659638348075\\
92.7272727272727	-0.0475819158237425\\
96.3636363636364	-0.110838199901011\\
100	-0.17364817766693\\
103.636363636364	-0.235758935509428\\
107.272727272727	-0.296920375328275\\
110.909090909091	-0.356886221591872\\
114.545454545455	-0.415415013001887\\
118.181818181818	-0.472271074772683\\
121.818181818182	-0.527225467610502\\
125.454545454545	-0.580056909571198\\
129.090909090909	-0.630552667084523\\
132.727272727273	-0.678509411557132\\
136.363636363636	-0.72373403810507\\
140	-0.766044443118978\\
143.636363636364	-0.805270257531059\\
147.272727272727	-0.841253532831181\\
150.909090909091	-0.873849377069785\\
154.545454545455	-0.902926538286621\\
158.181818181818	-0.928367933016073\\
161.818181818182	-0.950071117740945\\
165.454545454545	-0.967948701396356\\
169.090909090909	-0.981928697262707\\
172.727272727273	-0.991954812830795\\
176.363636363636	-0.997986676471884\\
180	-1\\
};

\addplot [color=blue,only marks,mark=o,mark options={solid}]
  table[row sep=crcr]{%
-180	-1\\
-152.307692307692	-0.885455991800417\\
-124.615384615385	-0.568064735308803\\
-96.9230769230769	-0.120536678161225\\
-69.2307692307692	0.354604872043075\\
-41.5384615384616	0.748510745341276\\
-13.8461538461539	0.970941787796838\\
13.8461538461538	0.970941787796839\\
41.5384615384615	0.748510745341276\\
69.2307692307692	0.354604872043075\\
96.9230769230769	-0.120536678161224\\
124.615384615385	-0.568064735308803\\
152.307692307692	-0.885455991800417\\
180	-1\\
};

\addplot [color=red,dashed]
  table[row sep=crcr]{%
-180	-0.999999977548887\\
-176.363636363636	-0.997986654065945\\
-172.727272727273	-0.991954790560195\\
-169.090909090909	-0.981928675217066\\
-165.454545454545	-0.96794867966439\\
-161.818181818182	-0.950071096410106\\
-158.181818181818	-0.928367912172195\\
-154.545454545455	-0.902926518013582\\
-150.909090909091	-0.873849357449161\\
-147.272727272727	-0.841253513941923\\
-143.636363636364	-0.805270239449171\\
-140	-0.766044425917215\\
-136.363636363636	-0.723734021852642\\
-132.727272727273	-0.678509396319426\\
-129.090909090909	-0.63055265292284\\
-125.454545454545	-0.580056896542508\\
-121.818181818182	-0.527225455767212\\
-118.181818181818	-0.472271064162425\\
-114.545454545455	-0.41541500366733\\
-110.909090909091	-0.356886213570548\\
-107.272727272727	-0.296920368652428\\
-103.636363636364	-0.235758930205883\\
-100	-0.173648173756988\\
-96.3636363636364	-0.110838197400361\\
-92.7272727272727	-0.047581914742397\\
-89.0909090909091	0.0158659634925489\\
-85.4545454545455	0.0792499550923586\\
-81.8181818181818	0.142314835093844\\
-78.1818181818182	0.204806663483597\\
-74.5454545454545	0.266473807724792\\
-70.9090909090909	0.327067955992604\\
-67.2727272727273	0.386345117038287\\
-63.6363636363636	0.444066602655814\\
-60	0.499999988795041\\
-56.3636363636364	0.553920051451326\\
-52.7272727272727	0.605609673563103\\
-49.0909090909091	0.654860719265656\\
-45.4545454545454	0.701474871980793\\
-41.8181818181818	0.745264432967703\\
-38.1818181818182	0.786053077119546\\
-34.5454545454545	0.823676562962418\\
-30.9090909090909	0.857983393997807\\
-27.2727272727273	0.888835428725515\\
-23.6363636363636	0.916108436890726\\
-20	0.939692599715398\\
-16.3636363636364	0.959492952099719\\
-12.7272727272727	0.975429765013048\\
-9.09090909090908	0.987438866534583\\
-5.45454545454543	0.995471900251033\\
-1.81818181818181	0.999496519970831\\
1.81818181818184	0.999496519970831\\
5.45454545454546	0.995471900251033\\
9.09090909090911	0.987438866534583\\
12.7272727272727	0.975429765013048\\
16.3636363636364	0.959492952099719\\
20	0.939692599715398\\
23.6363636363637	0.916108436890726\\
27.2727272727273	0.888835428725515\\
30.9090909090909	0.857983393997807\\
34.5454545454545	0.823676562962418\\
38.1818181818182	0.786053077119546\\
41.8181818181818	0.745264432967703\\
45.4545454545455	0.701474871980793\\
49.0909090909091	0.654860719265656\\
52.7272727272727	0.605609673563102\\
56.3636363636364	0.553920051451326\\
60	0.49999998879504\\
63.6363636363637	0.444066602655813\\
67.2727272727273	0.386345117038287\\
70.9090909090909	0.327067955992604\\
74.5454545454546	0.266473807724791\\
78.1818181818182	0.204806663483596\\
81.8181818181818	0.142314835093844\\
85.4545454545455	0.0792499550923586\\
89.0909090909091	0.0158659634925482\\
92.7272727272727	-0.0475819147423975\\
96.3636363636364	-0.110838197400361\\
100	-0.173648173756988\\
103.636363636364	-0.235758930205883\\
107.272727272727	-0.296920368652428\\
110.909090909091	-0.356886213570548\\
114.545454545455	-0.41541500366733\\
118.181818181818	-0.472271064162425\\
121.818181818182	-0.527225455767212\\
125.454545454545	-0.580056896542508\\
129.090909090909	-0.630552652922841\\
132.727272727273	-0.678509396319426\\
136.363636363636	-0.723734021852642\\
140	-0.766044425917215\\
143.636363636364	-0.805270239449171\\
147.272727272727	-0.841253513941923\\
150.909090909091	-0.873849357449161\\
154.545454545455	-0.902926518013582\\
158.181818181818	-0.928367912172196\\
161.818181818182	-0.950071096410106\\
165.454545454545	-0.96794867966439\\
169.090909090909	-0.981928675217067\\
172.727272727273	-0.991954790560195\\
176.363636363636	-0.997986654065945\\
180	-0.999999977548887\\
};

\node[right, align=left, text=black]
at (axis cs:-135,-0.7) {input: $\phi = \SI{0}{\arcdeg}$};
\node[right, align=left, text=black]
at (axis cs:-135,-0.9) {result: $\phi = \SI{0}{\arcdeg}$};
\end{axis}
\end{tikzpicture}%}
  \hfill
  \subfloat[$\cos\bigl(\theta-(+30)\bigr)$]{
  % This file was created by matlab2tikz.
%
%The latest updates can be retrieved from
%  http://www.mathworks.com/matlabcentral/fileexchange/22022-matlab2tikz-matlab2tikz
%where you can also make suggestions and rate matlab2tikz.
%
\begin{tikzpicture}

\begin{axis}[%
width=0.951\fwidth,
height=0.75\fwidth,
at={(0\fwidth,0\fwidth)},
scale only axis,
xmin=-180,
xmax=180,
xtick={-180,    0,  180},
xlabel={$\theta/\SI{}{\arcdeg}$},
xmajorgrids,
ymin=-0.999874127673875,
ymax=0.999874127673875,
ylabel={$\cos(\theta-\phi)$},
ymajorgrids,
axis background/.style={fill=white},
legend style={legend cell align=left,align=left,legend plot pos=left,draw=black}
]
\addplot [color=black,solid]
  table[row sep=crcr]{%
-180	-0.866025403784439\\
-176.363636363636	-0.895993774291336\\
-172.727272727273	-0.922354294104581\\
-169.090909090909	-0.945000818714669\\
-165.454545454545	-0.963842158559942\\
-161.818181818182	-0.978802446214779\\
-158.181818181818	-0.989821441880933\\
-154.545454545455	-0.996854775951942\\
-150.909090909091	-0.999874127673875\\
-147.272727272727	-0.998867339183008\\
-143.636363636364	-0.993838464461254\\
-140	-0.984807753012208\\
-136.363636363636	-0.971811568323542\\
-132.727272727273	-0.954902241444074\\
-129.090909090909	-0.934147860265107\\
-125.454545454545	-0.909631995354518\\
-121.818181818182	-0.881453363447582\\
-118.181818181818	-0.849725429949514\\
-114.545454545455	-0.814575952050336\\
-110.909090909091	-0.776146464291757\\
-107.272727272727	-0.734591708657533\\
-103.636363636364	-0.690079011482112\\
-100	-0.642787609686539\\
-96.3636363636364	-0.59290792905464\\
-92.7272727272727	-0.540640817455597\\
-89.0909090909091	-0.486196736100469\\
-85.4545454545455	-0.429794912089171\\
-81.8181818181818	-0.371662455660327\\
-78.1818181818182	-0.312033445698487\\
-74.5454545454545	-0.251147987181079\\
-70.9090909090909	-0.18925124436041\\
-67.2727272727273	-0.126592453573749\\
-63.6363636363636	-0.0634239196565642\\
-60	5.05319527541181e-016\\
-56.3636363636364	0.0634239196565648\\
-52.7272727272727	0.12659245357375\\
-49.0909090909091	0.18925124436041\\
-45.4545454545454	0.25114798718108\\
-41.8181818181818	0.312033445698487\\
-38.1818181818182	0.371662455660328\\
-34.5454545454545	0.429794912089172\\
-30.9090909090909	0.486196736100469\\
-27.2727272727273	0.540640817455598\\
-23.6363636363636	0.592907929054641\\
-20	0.642787609686539\\
-16.3636363636364	0.690079011482112\\
-12.7272727272727	0.734591708657533\\
-9.09090909090908	0.776146464291757\\
-5.45454545454543	0.814575952050336\\
-1.81818181818181	0.849725429949514\\
1.81818181818184	0.881453363447582\\
5.45454545454546	0.909631995354518\\
9.09090909090911	0.934147860265107\\
12.7272727272727	0.954902241444074\\
16.3636363636364	0.971811568323542\\
20	0.984807753012208\\
23.6363636363637	0.993838464461254\\
27.2727272727273	0.998867339183008\\
30.9090909090909	0.999874127673875\\
34.5454545454545	0.996854775951942\\
38.1818181818182	0.989821441880933\\
41.8181818181818	0.978802446214779\\
45.4545454545455	0.963842158559942\\
49.0909090909091	0.945000818714668\\
52.7272727272727	0.922354294104581\\
56.3636363636364	0.895993774291336\\
60	0.866025403784438\\
63.6363636363637	0.832569854634771\\
67.2727272727273	0.795761840530832\\
70.9090909090909	0.755749574354258\\
74.5454545454546	0.712694171378862\\
78.1818181818182	0.666769000516291\\
81.8181818181818	0.618158986220605\\
85.4545454545455	0.567059863862771\\
89.0909090909091	0.513677391573406\\
92.7272727272727	0.45822652172741\\
96.3636363636364	0.400930535406614\\
100	0.342020143325669\\
103.636363636364	0.281732556841429\\
107.272727272727	0.22031053278654\\
110.909090909091	0.15800139597335\\
114.545454545455	0.0950560433041819\\
118.181818181818	0.0317279334980671\\
121.818181818182	-0.0317279334980679\\
125.454545454545	-0.0950560433041827\\
129.090909090909	-0.158001395973351\\
132.727272727273	-0.220310532786541\\
136.363636363636	-0.28173255684143\\
140	-0.342020143325669\\
143.636363636364	-0.400930535406614\\
147.272727272727	-0.458226521727411\\
150.909090909091	-0.513677391573407\\
154.545454545455	-0.567059863862771\\
158.181818181818	-0.618158986220606\\
161.818181818182	-0.666769000516292\\
165.454545454545	-0.712694171378863\\
169.090909090909	-0.755749574354259\\
172.727272727273	-0.795761840530832\\
176.363636363636	-0.832569854634772\\
180	-0.866025403784439\\
};

\addplot [color=blue,only marks,mark=o,mark options={solid}]
  table[row sep=crcr]{%
-180	-0.866025403784439\\
-152.307692307692	-0.999188960281069\\
-124.615384615385	-0.903450415706285\\
-96.9230769230769	-0.600742256776532\\
-69.2307692307692	-0.160411274341826\\
-41.5384615384616	0.316667992325634\\
-13.8461538461539	0.721202425731447\\
13.8461538461538	0.960518081924436\\
41.5384615384615	0.979790648616705\\
69.2307692307692	0.774604929331893\\
96.9230769230769	0.391966606025713\\
124.615384615385	-0.080466567836728\\
152.307692307692	-0.534465805383544\\
180	-0.866025403784439\\
};

\addplot [color=red,dashed]
  table[row sep=crcr]{%
-180	-0.866025384341205\\
-176.363636363636	-0.89599375416788\\
-172.727272727273	-0.922354273381886\\
-169.090909090909	-0.945000797476129\\
-165.454545454545	-0.96384213689103\\
-161.818181818182	-0.9788024242027\\
-158.181818181818	-0.989821419614274\\
-154.545454545455	-0.996854753520316\\
-150.909090909091	-0.999874105167557\\
-147.272727272727	-0.998867316692576\\
-143.636363636364	-0.99383844207722\\
-140	-0.984807730824658\\
-136.363636363636	-0.971811546421768\\
-132.727272727273	-0.95490221991622\\
-129.090909090909	-0.93414783919781\\
-125.454545454545	-0.909631974832562\\
-121.818181818182	-0.881453343553552\\
-118.181818181818	-0.849725410763469\\
-114.545454545455	-0.814575933649482\\
-110.909090909091	-0.776146446750142\\
-107.272727272727	-0.734591692045743\\
-103.636363636364	-0.690078995866987\\
-100	-0.642787595130909\\
-96.3636363636364	-0.592907915617068\\
-92.7272727272727	-0.540640805190143\\
-89.0909090909091	-0.486196725056472\\
-85.4545454545455	-0.429794902311057\\
-81.8181818181818	-0.371662447187419\\
-78.1818181818182	-0.312033438564854\\
-74.5454545454545	-0.251147981415399\\
-70.9090909090909	-0.18925123998585\\
-67.2727272727273	-0.126592450607877\\
-63.6363636363636	-0.063423918111275\\
-60	1.18532823631039e-010\\
-56.3636363636364	0.0634239183479108\\
-52.7272727272727	0.126592450843227\\
-49.0909090909091	0.189251240219061\\
-45.4545454545454	0.251147981645628\\
-41.8181818181818	0.31203343879127\\
-38.1818181818182	0.371662447409206\\
-34.5454545454545	0.429794902527418\\
-30.9090909090909	0.486196725266632\\
-27.2727272727273	0.54064080539335\\
-23.6363636363636	0.592907915812601\\
-20	0.642787595318076\\
-16.3636363636364	0.69007899604513\\
-12.7272727272727	0.734591692214239\\
-9.09090909090908	0.776146446908411\\
-5.45454545454543	0.814575933796982\\
-1.81818181818181	0.8497254108997\\
1.81818181818184	0.881453343678062\\
5.45454545454546	0.909631974944945\\
9.09090909090911	0.934147839297711\\
12.7272727272727	0.954902220003332\\
16.3636363636364	0.971811546495835\\
20	0.984807730885477\\
23.6363636363637	0.993838442124644\\
27.2727272727273	0.998867316726508\\
30.9090909090909	0.999874105187957\\
34.5454545454545	0.996854753527197\\
38.1818181818182	0.989821419607705\\
41.8181818181818	0.978802424182802\\
45.4545454545455	0.96384213685798\\
49.0909090909091	0.945000797430155\\
52.7272727272727	0.92235427332327\\
56.3636363636364	0.895993754096953\\
60	0.866025384258347\\
63.6363636363637	0.832569835855645\\
67.2727272727273	0.795761822574335\\
70.9090909090909	0.755749557292743\\
74.5454545454546	0.712694155281078\\
78.1818181818182	0.666768985447105\\
81.8181818181818	0.618158972240744\\
85.4545454545455	0.567059851028574\\
89.0909090909091	0.5136773799366\\
92.7272727272727	0.4582265113349\\
96.3636363636364	0.400930526300295\\
100	0.342020135542257\\
103.636363636364	0.281732550412313\\
107.272727272727	0.220310527737656\\
110.909090909091	0.158001392325075\\
114.545454545455	0.095056041071255\\
118.181818181818	0.031727932689527\\
121.818181818182	-0.0317279328789177\\
125.454545454545	-0.0950560412597873\\
129.090909090909	-0.158001392511894\\
132.727272727273	-0.220310527921914\\
136.363636363636	-0.281732550593172\\
140	-0.342020135718892\\
143.636363636364	-0.400930526471901\\
147.272727272727	-0.458226511500689\\
150.909090909091	-0.513677380095808\\
154.545454545455	-0.567059851180465\\
158.181818181818	-0.618158972384611\\
161.818181818182	-0.666768985582273\\
165.454545454545	-0.712694155406906\\
169.090909090909	-0.75574955740863\\
172.727272727273	-0.795761822679717\\
176.363636363636	-0.832569835950002\\
180	-0.866025384341205\\
};

\node[right, align=left, text=black]
at (axis cs:-135,-0.7) {input: $\phi = \SI{30}{\arcdeg}$};
\node[right, align=left, text=black]
at (axis cs:-135,-0.9) {result: $\phi = \SI{30}{\arcdeg}$};
\end{axis}
\end{tikzpicture}%}
  \caption{Base function}
  \label{fig:base}
\end{figure}

It is very important to mention that any initial phase shift (or offset) should be accounted for. Typically a function will be defined between $\theta_1$ and $\theta_2$ where $\theta_1\neq 0$. The difference between $\theta_1$ and $0$ should be accounted for since the phase from the DFT assumes that the sampled data has it origin at $(0,0)$. Neglecting this may leads to inaccurate results. 

The DFT of a vector $x$ of length $N$ is another vector $X$ of length $N$. This notation uses $j$ and $k$ for indices that run from 1 to $N$. The vector $x$ is sampled at sampling wave length $d\lambda=\lambda_s$ and is a multiple of the fundamental (or the working harmonic), i.e.~$\lambda_1 = N\lambda_s$. The basic definition of the DFT\footnote{See \cite{REF-01048, REF-01049, REF-01050} for some reasons why the exponent is negative.} and its inverse are given as
\begin{equation}\label{eqn:dft}
  \begin{aligned}
  X(k) &= \sum_{j=1}^N \; x(j) \; \omega_N^{(j-1)(k-1)}\\
  x(j) &= \cfrac{1}{N}\sum_{k=1}^N \; X(k) \; \omega_N^{-(j-1)(k-1)}
  \end{aligned}
  \qquad
  \begin{cases}
    j \in \left\{1,\ldots,N  \right\} \rightarrow \; \mbox{sampled data}\\
  	k \in \left\{1,\ldots,N  \right\} \rightarrow \; \mbox{harmonic order}\\
    \omega_N = e^{-\theta i}  \\
  	\theta = \frac{2\pi}{N}   \\
  	e^{-\theta i} = \cos (-\theta) + i \sin (-\theta)
  \end{cases}
\end{equation}
The user defined function is given at the end of this section. An example that verify the function is given in Fig.~\ref{fig:user}. This basic function, for example, can now be used to determine the direct and quadrature axes in an electrical machine. 

\begin{figure}[htbp]
  \centering  
  \fontsize{6}{0}\selectfont
  \setlength\fwidth{0.43\textwidth}
  \subfloat[$\phi = \SI{-60}{\arcdeg}$]{
  % This file was created by matlab2tikz.
%
%The latest updates can be retrieved from
%  http://www.mathworks.com/matlabcentral/fileexchange/22022-matlab2tikz-matlab2tikz
%where you can also make suggestions and rate matlab2tikz.
%
\begin{tikzpicture}

\begin{axis}[%
width=0.951\fwidth,
height=0.75\fwidth,
at={(0\fwidth,0\fwidth)},
scale only axis,
xmin=-180,
xmax=180,
xtick={-180,    0,  180},
xlabel={$\theta/\SI{}{\arcdeg}$},
xmajorgrids,
ymin=-0.999496542383185,
ymax=1,
ylabel={$\cos(\theta-\phi)$},
ymajorgrids,
axis background/.style={fill=white},
legend style={legend cell align=left,align=left,legend plot pos=left,draw=black}
]
\addplot [color=black,solid]
  table[row sep=crcr]{%
-180	-0.5\\
-176.363636363636	-0.444066612605774\\
-172.727272727273	-0.386345125693129\\
-169.090909090909	-0.327067963317421\\
-165.454545454545	-0.266473813690035\\
-161.818181818182	-0.204806668065191\\
-158.181818181818	-0.142314838273285\\
-154.545454545455	-0.0792499568567883\\
-150.909090909091	-0.015865963834808\\
-147.272727272727	0.0475819158237424\\
-143.636363636364	0.110838199901011\\
-140	0.17364817766693\\
-136.363636363636	0.235758935509427\\
-132.727272727273	0.296920375328275\\
-129.090909090909	0.356886221591872\\
-125.454545454545	0.415415013001886\\
-121.818181818182	0.472271074772683\\
-118.181818181818	0.527225467610502\\
-114.545454545455	0.580056909571198\\
-110.909090909091	0.630552667084523\\
-107.272727272727	0.678509411557132\\
-103.636363636364	0.72373403810507\\
-100	0.766044443118978\\
-96.3636363636364	0.805270257531059\\
-92.7272727272727	0.841253532831181\\
-89.0909090909091	0.873849377069785\\
-85.4545454545455	0.902926538286621\\
-81.8181818181818	0.928367933016073\\
-78.1818181818182	0.950071117740945\\
-74.5454545454545	0.967948701396356\\
-70.9090909090909	0.981928697262707\\
-67.2727272727273	0.991954812830795\\
-63.6363636363636	0.997986676471884\\
-60	1\\
-56.3636363636364	0.997986676471884\\
-52.7272727272727	0.991954812830795\\
-49.0909090909091	0.981928697262707\\
-45.4545454545454	0.967948701396356\\
-41.8181818181818	0.950071117740945\\
-38.1818181818182	0.928367933016073\\
-34.5454545454545	0.902926538286621\\
-30.9090909090909	0.873849377069785\\
-27.2727272727273	0.841253532831181\\
-23.6363636363636	0.805270257531059\\
-20	0.766044443118978\\
-16.3636363636364	0.72373403810507\\
-12.7272727272727	0.678509411557132\\
-9.09090909090908	0.630552667084523\\
-5.45454545454543	0.580056909571198\\
-1.81818181818181	0.527225467610502\\
1.81818181818184	0.472271074772683\\
5.45454545454546	0.415415013001886\\
9.09090909090911	0.356886221591872\\
12.7272727272727	0.296920375328275\\
16.3636363636364	0.235758935509427\\
20	0.17364817766693\\
23.6363636363637	0.110838199901011\\
27.2727272727273	0.0475819158237424\\
30.9090909090909	-0.015865963834808\\
34.5454545454545	-0.0792499568567883\\
38.1818181818182	-0.142314838273285\\
41.8181818181818	-0.204806668065191\\
45.4545454545455	-0.266473813690035\\
49.0909090909091	-0.327067963317421\\
52.7272727272727	-0.386345125693129\\
56.3636363636364	-0.444066612605774\\
60	-0.5\\
63.6363636363637	-0.55392006386611\\
67.2727272727273	-0.605609687137666\\
70.9090909090909	-0.654860733945285\\
74.5454545454546	-0.701474887706321\\
78.1818181818182	-0.745264449675755\\
81.8181818181818	-0.786053094742787\\
85.4545454545455	-0.823676581429832\\
89.0909090909091	-0.857983413234977\\
92.7272727272727	-0.888835448654923\\
96.3636363636364	-0.916108457432069\\
100	-0.939692620785908\\
103.636363636364	-0.959492973614497\\
107.272727272727	-0.975429786885407\\
110.909090909091	-0.987438888676394\\
114.545454545455	-0.995471922573085\\
118.181818181818	-0.999496542383185\\
121.818181818182	-0.999496542383185\\
125.454545454545	-0.995471922573085\\
129.090909090909	-0.987438888676394\\
132.727272727273	-0.975429786885407\\
136.363636363636	-0.959492973614497\\
140	-0.939692620785909\\
143.636363636364	-0.91610845743207\\
147.272727272727	-0.888835448654923\\
150.909090909091	-0.857983413234977\\
154.545454545455	-0.823676581429833\\
158.181818181818	-0.786053094742787\\
161.818181818182	-0.745264449675755\\
165.454545454545	-0.701474887706322\\
169.090909090909	-0.654860733945285\\
172.727272727273	-0.605609687137667\\
176.363636363636	-0.55392006386611\\
180	-0.5\\
};
\addlegendentry{reference};

\addplot [color=blue,only marks,mark=o,mark options={solid}]
  table[row sep=crcr]{%
-180	-0.5\\
-152.307692307692	-0.0402659380321012\\
-124.615384615385	0.428692551504903\\
-96.9230769230769	0.799442754668093\\
-69.2307692307692	0.987050221353063\\
-41.5384615384616	0.948536438843722\\
-13.8461538461539	0.692724331684891\\
13.8461538461538	0.278217456111948\\
41.5384615384615	-0.200025693502446\\
69.2307692307692	-0.632445349309987\\
96.9230769230769	-0.919979432829317\\
124.615384615385	-0.996757286813706\\
152.307692307692	-0.845190053768316\\
180	-0.5\\
};
\addlegendentry{sampled data};

\addplot [color=red,dashed]
  table[row sep=crcr]{%
-180	-0.499999988774443\\
-176.363636363636	-0.444066602648743\\
-172.727272727273	-0.386345117044689\\
-169.090909090909	-0.32706795601237\\
-165.454545454545	-0.266473807757759\\
-161.818181818182	-0.20480666352955\\
-158.181818181818	-0.142314835152515\\
-154.545454545455	-0.0792499551634276\\
-150.909090909091	-0.0158659635756475\\
-147.272727272727	0.0475819146476868\\
-143.636363636364	0.110838197294502\\
-140	0.173648173640492\\
-136.363636363636	0.2357589300793\\
-132.727272727273	0.296920368516351\\
-129.090909090909	0.356886213425609\\
-125.454545454545	0.415415003514194\\
-121.818181818182	0.472271064001793\\
-118.181818181818	0.527225455599813\\
-114.545454545455	0.5800568963691\\
-110.909090909091	0.630552652744203\\
-107.272727272727	0.678509396136362\\
-103.636363636364	0.723734021665971\\
-100	0.766044425727773\\
-96.3636363636364	0.805270239257803\\
-92.7272727272727	0.841253513749482\\
-89.0909090909091	0.873849357256506\\
-85.4545454545455	0.902926517821571\\
-81.8181818181818	0.928367911981684\\
-78.1818181818182	0.950071096221944\\
-74.5454545454545	0.96794867947942\\
-70.9090909090909	0.981928675036114\\
-67.2727272727273	0.991954790384073\\
-63.6363636363636	0.997986653895445\\
-60	0.999999977384778\\
-56.3636363636364	0.997986653908972\\
-52.7272727272727	0.991954790411072\\
-49.0909090909091	0.981928675076477\\
-45.4545454545454	0.967948679532984\\
-41.8181818181818	0.950071096288495\\
-38.1818181818182	0.928367912060952\\
-34.5454545454545	0.902926517913237\\
-30.9090909090909	0.873849357360202\\
-27.2727272727273	0.84125351386479\\
-23.6363636363636	0.805270239384258\\
-20	0.766044425864866\\
-16.3636363636364	0.723734021813151\\
-12.7272727272727	0.678509396293036\\
-9.09090909090908	0.630552652909739\\
-5.45454545454543	0.580056896542832\\
-1.81818181818181	0.527225455781042\\
1.81818181818184	0.472271064189789\\
5.45454545454546	0.4154150037082\\
9.09090909090911	0.356886213624843\\
12.7272727272727	0.296920368720013\\
16.3636363636364	0.235758930286567\\
20	0.173648173850532\\
23.6363636363637	0.110838197506468\\
27.2727272727273	0.0475819148607249\\
30.9090909090909	-0.0158659633623947\\
34.5454545454545	-0.0792499549508188\\
38.1818181818182	-0.142314834941406\\
41.8181818181818	-0.204806663320791\\
45.4545454545455	-0.266473807552192\\
49.0909090909091	-0.327067955810821\\
52.7272727272727	-0.38634511684797\\
56.3636363636364	-0.444066602457646\\
60	-0.499999988589738\\
63.6363636363637	-0.553920051239632\\
67.2727272727273	-0.605609673345786\\
70.9090909090909	-0.65486071904351\\
74.5454545454546	-0.701474871754628\\
78.1818181818182	-0.745264432738348\\
81.8181818181818	-0.78605307688784\\
85.4545454545455	-0.823676562729212\\
89.0909090909091	-0.857983393763957\\
92.7272727272727	-0.888835428491879\\
96.3636363636364	-0.916108436658163\\
100	-0.939692599484761\\
103.636363636364	-0.959492951871854\\
107.272727272727	-0.97542976478879\\
110.909090909091	-0.987438866314751\\
114.545454545455	-0.995471900036429\\
118.181818181818	-0.999496519762238\\
121.818181818182	-0.999496519769005\\
125.454545454545	-0.995471900056703\\
129.090909090909	-0.987438866348449\\
132.727272727273	-0.975429764835777\\
136.363636363636	-0.959492951931942\\
140	-0.939692599557707\\
143.636363636364	-0.916108436743673\\
147.272727272727	-0.88883542858961\\
150.909090909091	-0.857983393873514\\
154.545454545455	-0.823676562850155\\
158.181818181818	-0.78605307701968\\
161.818181818182	-0.745264432880556\\
165.454545454545	-0.701474871906631\\
169.090909090909	-0.654860719204695\\
172.727272727273	-0.605609673515505\\
176.363636363636	-0.553920051417201\\
180	-0.499999988774443\\
};
\addlegendentry{result};

\node[right, align=left, text=black]
at (axis cs:-135,-0.7) {input: $\phi = \SI{-60}{\arcdeg}$};
\node[right, align=left, text=black]
at (axis cs:-135,-0.9) {result: $\phi = \SI{-60}{\arcdeg}$};
\end{axis}
\end{tikzpicture}%}
  \hfill
  \subfloat[$\phi = \SI{33}{\arcdeg}$]{
  % This file was created by matlab2tikz.
%
%The latest updates can be retrieved from
%  http://www.mathworks.com/matlabcentral/fileexchange/22022-matlab2tikz-matlab2tikz
%where you can also make suggestions and rate matlab2tikz.
%
\begin{tikzpicture}

\begin{axis}[%
width=0.951\fwidth,
height=0.75\fwidth,
at={(0\fwidth,0\fwidth)},
scale only axis,
xmin=-180,
xmax=180,
xtick={-180,    0,  180},
xlabel={$\theta/\SI{}{\arcdeg}$},
xmajorgrids,
ymin=-0.999988671274373,
ymax=0.999636243401075,
ylabel={$\cos(\theta-\phi)$},
ymajorgrids,
axis background/.style={fill=white},
legend style={legend cell align=left,align=left,legend plot pos=left,draw=black}
]
\addplot [color=black,solid]
  table[row sep=crcr]{%
-180	-0.838670567945424\\
-176.363636363636	-0.871525195157263\\
-172.727272727273	-0.900870498007591\\
-169.090909090909	-0.926588313319072\\
-165.454545454545	-0.948575084526387\\
-161.818181818182	-0.96674227866198\\
-158.181818181818	-0.981016742847065\\
-154.545454545455	-0.991340998852451\\
-150.909090909091	-0.997673474543087\\
-147.272727272727	-0.999988671274373\\
-143.636363636364	-0.998277266566208\\
-140	-0.992546151641322\\
-136.363636363636	-0.982818403676756\\
-132.727272727273	-0.969133192880214\\
-129.090909090909	-0.951545624765466\\
-125.454545454545	-0.930126518261886\\
-121.818181818182	-0.904962120551624\\
-118.181818181818	-0.876153759782643\\
-114.545454545455	-0.843817437056026\\
-110.909090909091	-0.808083359330492\\
-107.272727272727	-0.769095415124919\\
-103.636363636364	-0.727010595130074\\
-100	-0.681998360062498\\
-96.3636363636364	-0.634239958306023\\
-92.7272727272727	-0.58392769608849\\
-89.0909090909091	-0.531264163132451\\
-85.4545454545455	-0.476461416897854\\
-81.8181818181818	-0.419740128701496\\
-78.1818181818182	-0.361328695151522\\
-74.5454545454545	-0.301462318474883\\
-70.9090909090909	-0.240382059440992\\
-67.2727272727273	-0.178333866695083\\
-63.6363636363636	-0.11556758640982\\
-60	-0.0523359562429436\\
-56.3636363636364	0.011106412348073\\
-52.7272727272727	0.0745040593365034\\
-49.0909090909091	0.137601704773729\\
-45.4545454545454	0.200145276711495\\
-41.8181818181818	0.261882934259972\\
-38.1818181818182	0.322566081662134\\
-34.5454545454545	0.381950369301131\\
-30.9090909090909	0.439796677609955\\
-27.2727272727273	0.495872079921541\\
-23.6363636363636	0.549950780382244\\
-20	0.601815023152048\\
-16.3636363636364	0.651255969230482\\
-12.7272727272727	0.69807453737756\\
-9.09090909090908	0.742082205743678\\
-5.45454545454543	0.783101770980556\\
-1.81818181818181	0.820968061776585\\
1.81818181818184	0.855528603943402\\
5.45454545454546	0.886644234375629\\
9.09090909090911	0.914189661411583\\
12.7272727272727	0.938053969338576\\
16.3636363636364	0.958141065011347\\
20	0.974370064785235\\
23.6363636363637	0.986675620206076\\
27.2727272727273	0.99500818114536\\
30.9090909090909	0.999334195321108\\
34.5454545454545	0.999636243401075\\
38.1818181818182	0.99591310914425\\
41.8181818181818	0.988179784298226\\
45.4545454545455	0.976467408232731\\
49.0909090909091	0.96082314255237\\
52.7272727272727	0.941309981193492\\
56.3636363636364	0.91800649676984\\
60	0.891006524188368\\
63.6363636363637	0.86041878280919\\
67.2727272727273	0.826366438671088\\
70.9090909090909	0.788986608545343\\
74.5454545454546	0.748429807814892\\
78.1818181818182	0.704859344402008\\
81.8181818181818	0.658450661184931\\
85.4545454545455	0.60939062955132\\
89.0909090909091	0.557876796933131\\
92.7272727272727	0.504116591352832\\
96.3636363636364	0.448326486183965\\
100	0.390731128489274\\
103.636363636364	0.331562434446274\\
107.272727272727	0.271058655502654\\
110.909090909091	0.209463419021789\\
114.545454545455	0.14702474728133\\
118.181818181818	0.0839940587750392\\
121.818181818182	0.0206251558392407\\
125.454545454545	-0.0428267973196022\\
129.090909090909	-0.106106302081091\\
132.727272727273	-0.168958554213657\\
136.363636363636	-0.231130469881273\\
140	-0.292371704722736\\
143.636363636364	-0.352435661900054\\
147.272727272727	-0.411080485056869\\
150.909090909091	-0.468070032188656\\
154.545454545455	-0.523174826503221\\
158.181818181818	-0.576172980442753\\
161.818181818182	-0.626851089146705\\
165.454545454545	-0.675005089757848\\
169.090909090909	-0.720441083111378\\
172.727272727273	-0.762976114498409\\
176.363636363636	-0.802438910360019\\
180	-0.838670567945424\\
};
\addlegendentry{reference};

\addplot [color=blue,only marks,mark=o,mark options={solid}]
  table[row sep=crcr]{%
-180	-0.838670567945424\\
-152.307692307692	-0.995712250166461\\
-124.615384615385	-0.92464830292783\\
-96.9230769230769	-0.641758561419209\\
-69.2307692307692	-0.211849653459688\\
-41.5384615384616	0.26659144828965\\
-13.8461538461539	0.683959652560582\\
13.8461538461538	0.944640948666457\\
41.5384615384615	0.988916415527591\\
69.2307692307692	0.806642992324848\\
96.9230769230769	0.439577432755751\\
124.615384615385	-0.0281900454545712\\
152.307692307692	-0.489499508901407\\
180	-0.838670567945424\\
};
\addlegendentry{sampled data};

\addplot [color=red,dashed]
  table[row sep=crcr]{%
-180	-0.838670549116336\\
-176.363636363636	-0.871525175582497\\
-172.727272727273	-0.90087047776592\\
-169.090909090909	-0.926588292491956\\
-165.454545454545	-0.948575063197643\\
-161.818181818182	-0.966742256917446\\
-158.181818181818	-0.981016720774251\\
-154.545454545455	-0.99134097654019\\
-150.909090909091	-0.997673452081176\\
-147.272727272727	-0.999988648753213\\
-143.636363636364	-0.998277244076437\\
-140	-0.992546129273451\\
-136.363636363636	-0.982818381520807\\
-132.727272727273	-0.969133171025355\\
-129.090909090909	-0.951545603299652\\
-125.454545454545	-0.930126497271506\\
-121.818181818182	-0.904962100121153\\
-118.181818181818	-0.8761537399943\\
-114.545454545455	-0.843817417989446\\
-110.909090909091	-0.808083341062403\\
-107.272727272727	-0.769095397728834\\
-103.636363636364	-0.727010578675993\\
-100	-0.681998344616631\\
-96.3636363636364	-0.634239943930518\\
-92.7272727272727	-0.583927682841187\\
-89.0909090909091	-0.531264151066644\\
-85.4545454545455	-0.476461406062082\\
-81.8181818181818	-0.419740119139345\\
-78.1818181818182	-0.361328686901448\\
-74.5454545454545	-0.30146231157006\\
-70.9090909090909	-0.240382053909178\\
-67.2727272727273	-0.178333862558506\\
-63.6363636363636	-0.115567583685089\\
-60	-0.0523359549409847\\
-56.3636363636364	0.0111064122220639\\
-52.7272727272727	0.07450405778308\\
-49.0909090909091	0.137601701799193\\
-45.4545454545454	0.20014527232787\\
-41.8181818181818	0.261882928484955\\
-38.1818181818182	0.322566074519026\\
-34.5454545454545	0.381950360818741\\
-30.9090909090909	0.439796667822486\\
-27.2727272727273	0.495872068868449\\
-23.6363636363636	0.549950768108082\\
-20	0.601815009706287\\
-16.3636363636364	0.651255954667309\\
-12.7272727272727	0.698074521755662\\
-9.09090909090908	0.742082189126005\\
-5.45454545454543	0.783101753434069\\
-1.81818181818181	0.820968043371983\\
1.81818181818184	0.855528584754841\\
5.45454545454546	0.88664421448042\\
9.09090909090911	0.914189640889884\\
12.7272727272727	0.938053948273067\\
16.3636363636364	0.958141043486898\\
20	0.974370042888563\\
23.6363636363637	0.986675598025398\\
27.2727272727273	0.995008158770035\\
30.9090909090909	0.999334172841282\\
34.5454545454545	0.999636220907312\\
38.1818181818182	0.995913086727169\\
41.8181818181818	0.988179762048141\\
45.4545454545455	0.976467386239281\\
49.0909090909091	0.960823120904162\\
52.7272727272727	0.941309959977741\\
56.3636363636364	0.91800647607202\\
60	0.89100650409187\\
63.6363636363637	0.860418763394981\\
67.2727272727273	0.826366420017389\\
70.9090909090909	0.788986590727312\\
74.5454545454546	0.748429790904322\\
78.1818181818182	0.704859328467039\\
81.8181818181818	0.658450646289773\\
85.4545454545455	0.609390615755997\\
89.0909090909091	0.557876784293239\\
92.7272727272727	0.504116579919313\\
96.3636363636364	0.448326476002904\\
100	0.390731119601714\\
103.636363636364	0.331562426888047\\
107.272727272727	0.271058649304242\\
110.909090909091	0.209463414208196\\
114.545454545455	0.147024743871986\\
118.181818181818	0.0839940567837184\\
121.818181818182	0.0206251552740078\\
125.454545454545	-0.0428267964564247\\
129.090909090909	-0.106106299792933\\
132.727272727273	-0.168958550509685\\
136.363636363636	-0.231130464776355\\
140	-0.292371698237383\\
143.636363636364	-0.352435654060331\\
147.272727272727	-0.4110804758943\\
150.909090909091	-0.468070021740088\\
154.545454545455	-0.523174814810681\\
158.181818181818	-0.576172967553276\\
161.818181818182	-0.626851075112145\\
165.454545454545	-0.675005074634672\\
169.090909090909	-0.720441066960435\\
172.727272727273	-0.762976097384686\\
176.363636363636	-0.802438892352383\\
180	-0.838670549116336\\
};
\addlegendentry{result};

\node[right, align=left, text=black]
at (axis cs:-135,-0.7) {input: $\phi = \SI{33}{\arcdeg}$};
\node[right, align=left, text=black]
at (axis cs:-135,-0.9) {result: phase = $\phi = \SI{33}{\arcdeg}$};
\end{axis}
\end{tikzpicture}%}
  \caption{Test examples}
  \label{fig:user}
\end{figure}

\verbatiminput{../code/test_dft.m}

\chapter{Slot assignment \texttt{CDesign.m}}\label{sec:malg}
\verbatiminput{../code/CDesign.m}

\chapter{Examples \texttt{arun.m}}\label{sec:mex}
\verbatiminput{../code/arun.m}

\begin{figure}[htbp]
\centering
\fontsize{8}{0}\selectfont
\setlength\fwidth{0.45\textwidth}
\subfloat{
% This file was created by matlab2tikz.
%
%The latest updates can be retrieved from
%  http://www.mathworks.com/matlabcentral/fileexchange/22022-matlab2tikz-matlab2tikz
%where you can also make suggestions and rate matlab2tikz.
%
\definecolor{mycolor1}{rgb}{1.00000,1.00000,0.00000}%
%
\begin{tikzpicture}

\begin{axis}[%
width=0.951\fwidth,
height=0.75\fwidth,
at={(0\fwidth,0\fwidth)},
scale only axis,
xmin=0,
xmax=72,
xtick={0,6,12,18,24,30,36,42,48,54,60,66},
xticklabels={{ },{1},{ },{2},{ },{3},{ },{4},{ },{5},{ },{6},{ }},
xlabel={Slot number},
xmajorgrids,
ymin=-1.75,
ymax=1,
ytick={-1.75,-1.5,-1.25,-1,-0.75,-0.5,-0.25,0,0.25,0.5,0.75,1},
yticklabels={{ },{ },{ },{-1.0},{ },{-0.5},{ },{ },{ },{0.5},{ },{1.0}},
ymajorgrids,
axis background/.style={fill=white}
]
\addplot [color=blue,solid,forget plot]
  table[row sep=crcr]{%
0	-1.75\\
3	-1.75\\
3	-1.25\\
9	-1.25\\
9	-1.75\\
12	-1.75\\
};

\addplot[area legend,solid,draw=black,fill=red,forget plot]
table[row sep=crcr] {%
x	y\\
3	-1.75\\
3	-1.25\\
9	-1.25\\
9	-1.75\\
3	-1.75\\
}--cycle;
\addplot [color=blue,solid,forget plot]
  table[row sep=crcr]{%
12	-1.75\\
15	-1.75\\
15	-1.25\\
21	-1.25\\
21	-1.75\\
24	-1.75\\
};

\addplot[area legend,solid,draw=black,fill=red,forget plot]
table[row sep=crcr] {%
x	y\\
15	-1.75\\
15	-1.25\\
21	-1.25\\
21	-1.75\\
15	-1.75\\
}--cycle;
\addplot [color=blue,solid,forget plot]
  table[row sep=crcr]{%
24	-1.75\\
27	-1.75\\
27	-1.25\\
33	-1.25\\
33	-1.75\\
36	-1.75\\
};

\addplot[area legend,solid,draw=black,fill=blue,forget plot]
table[row sep=crcr] {%
x	y\\
27	-1.75\\
27	-1.25\\
33	-1.25\\
33	-1.75\\
27	-1.75\\
}--cycle;
\addplot [color=blue,solid,forget plot]
  table[row sep=crcr]{%
36	-1.75\\
39	-1.75\\
39	-1.25\\
45	-1.25\\
45	-1.75\\
48	-1.75\\
};

\addplot[area legend,solid,draw=black,fill=blue,forget plot]
table[row sep=crcr] {%
x	y\\
39	-1.75\\
39	-1.25\\
45	-1.25\\
45	-1.75\\
39	-1.75\\
}--cycle;
\addplot [color=blue,solid,forget plot]
  table[row sep=crcr]{%
48	-1.75\\
51	-1.75\\
51	-1.25\\
57	-1.25\\
57	-1.75\\
60	-1.75\\
};

\addplot[area legend,solid,draw=black,fill=mycolor1,forget plot]
table[row sep=crcr] {%
x	y\\
51	-1.75\\
51	-1.25\\
57	-1.25\\
57	-1.75\\
51	-1.75\\
}--cycle;
\addplot [color=blue,solid,forget plot]
  table[row sep=crcr]{%
60	-1.75\\
63	-1.75\\
63	-1.25\\
69	-1.25\\
69	-1.75\\
72	-1.75\\
};

\addplot[area legend,solid,draw=black,fill=mycolor1,forget plot]
table[row sep=crcr] {%
x	y\\
63	-1.75\\
63	-1.25\\
69	-1.25\\
69	-1.75\\
63	-1.75\\
}--cycle;

\addplot[area legend,solid,table/row sep=crcr,patch,patch type=rectangle,fill=black,faceted color=black,forget plot,patch table={%
0	1	2	3\\
4	5	6	7\\
8	9	10	11\\
12	13	14	15\\
16	17	18	19\\
20	21	22	23\\
}]
table[row sep=crcr] {%
x	y\\
3	0\\
3	1\\
9	1\\
9	0\\
15	0\\
15	-1\\
21	-1\\
21	0\\
27	0\\
27	-0.5\\
33	-0.5\\
33	0\\
39	0\\
39	0.5\\
45	0.5\\
45	0\\
51	0\\
51	-0.5\\
57	-0.5\\
57	0\\
63	0\\
63	0.5\\
69	0.5\\
69	0\\
};
\addplot [color=black,solid,forget plot]
  table[row sep=crcr]{%
0	0\\
72	0\\
};
\addplot [color=black,dashed,forget plot]
  table[row sep=crcr]{%
0	0.75\\
0.361809045226131	0.775830679917935\\
0.723618090452261	0.798568671165863\\
1.08542713567839	0.818123333440775\\
1.44723618090452	0.834416716122736\\
1.80904522613065	0.847383869008925\\
2.17085427135678	0.856973101224192\\
2.53266331658291	0.863146187276048\\
2.89447236180905	0.865878519432698\\
3.25628140703518	0.86515920581667\\
3.61809045226131	0.86099111382304\\
3.97989949748744	0.853390858689149\\
4.34170854271357	0.842388737261398\\
4.7035175879397	0.828028607223126\\
5.06532663316583	0.810367712265025\\
5.42713567839196	0.789476453894991\\
5.78894472361809	0.765438110797066\\
6.15075376884422	0.738348506858175\\
6.51256281407035	0.708315629186001\\
6.87437185929648	0.675459197640701\\
7.23618090452261	0.639910187596421\\
7.59798994974874	0.60181030783504\\
7.95979899497487	0.561311435653389\\
8.321608040201	0.518575011435804\\
8.68341708542714	0.47377139510539\\
9.04522613065327	0.427079187019401\\
9.4070351758794	0.378684516015819\\
9.76884422110553	0.328780297449213\\
10.1306532663317	0.277565464173544\\
10.4924623115578	0.225244173537455\\
10.8542713567839	0.172024993553193\\
11.21608040201	0.118120071483329\\
11.5778894472362	0.0637442881595448\\
11.9396984924623	0.0091144014045972\\
12.3015075376884	-0.0455518180279412\\
12.6633165829146	-0.100036454551784\\
13.0251256281407	-0.154122316423391\\
13.3869346733668	-0.207593801531953\\
13.748743718593	-0.260237756852663\\
14.1105527638191	-0.311844328137193\\
14.4723618090452	-0.362207796454283\\
14.8341708542714	-0.411127398245754\\
15.1959798994975	-0.458408125628934\\
15.5577889447236	-0.503861503755286\\
15.9195979899497	-0.547306342126441\\
16.2814070351759	-0.588569456872673\\
16.643216080402	-0.627486361114604\\
17.0050251256281	-0.663901920656156\\
17.3668341708543	-0.697670972394967\\
17.7286432160804	-0.728658902985092\\
18.0904522613065	-0.756742185445292\\
18.4522613065327	-0.781808871573815\\
18.8140703517588	-0.803759038206766\\
19.1758793969849	-0.822505185541148\\
19.5376884422111	-0.837972585934749\\
19.8994974874372	-0.850099581792438\\
20.2613065326633	-0.858837831351431\\
20.6231155778894	-0.864152501385732\\
20.9849246231156	-0.866022406061596\\
21.3467336683417	-0.864440091390472\\
21.7085427135678	-0.859411864942792\\
22.070351758794	-0.85095777070414\\
22.4321608040201	-0.83911150917405\\
22.7939698492462	-0.823920303025929\\
23.1557788944724	-0.805444708863625\\
23.5175879396985	-0.783758375825046\\
23.8793969849246	-0.758947751995087\\
24.2412060301508	-0.731111739798208\\
24.6030150753769	-0.70036130174436\\
24.964824120603	-0.666819018099867\\
25.3266331658291	-0.630618598246528\\
25.6884422110553	-0.591904347676803\\
26.0502512562814	-0.550830592749784\\
26.4120603015075	-0.507561065501064\\
26.7738693467337	-0.462268250958809\\
27.1356783919598	-0.415132699567845\\
27.4974874371859	-0.366342307462617\\
27.8592964824121	-0.316091567458085\\
28.2211055276382	-0.264580793744308\\
28.5829145728643	-0.212015323375327\\
28.9447236180905	-0.15860469773543\\
29.3065326633166	-0.104561827245731\\
29.6683417085427	-0.050102142640777\\
30.0301507537688	0.004557263801557\\
30.391959798995	0.059198503653563\\
30.7537688442211	0.113603760904894\\
31.1155778894472	0.16755616024107\\
31.4773869346734	0.220840631572101\\
31.8391959798995	0.273244767364955\\
32.2010050251256	0.324559669362288\\
32.5628140703518	0.374580781312179\\
32.9246231155779	0.423108704389352\\
33.286432160804	0.469949992057386\\
33.6482412060301	0.514917921203351\\
34.0100502512563	0.557833236470916\\
34.3718592964824	0.598524864824771\\
34.7336683417085	0.636830597497939\\
35.0954773869347	0.672597736603494\\
35.4572864321608	0.705683703833137\\
35.8190954773869	0.735956608816151\\
36.1809045226131	0.76329577487311\\
36.5427135678392	0.787592220068522\\
36.9045226130653	0.808749091644792\\
37.2663316582915	0.826682052105722\\
37.6281407035176	0.841319615410515\\
37.9899497487437	0.852603431938098\\
38.3517587939699	0.860488521085832\\
38.713567839196	0.864943450575385\\
39.0753768844221	0.865950461751016\\
39.4371859296482	0.863505540370785\\
39.7989949748744	0.8576184326085\\
40.1608040201005	0.848312606202609\\
40.5226130653266	0.835625156906907\\
40.8844221105528	0.819606660615987\\
41.2462311557789	0.800320971754882\\
41.608040201005	0.777844968736597\\
41.9698492462312	0.75226824750221\\
42.3316582914573	0.723692764365157\\
42.6934673366834	0.692232429583471\\
43.0552763819095	0.658012653280072\\
43.4170854271357	0.621169845521232\\
43.7788944723618	0.581850872546036\\
44.1407035175879	0.540212471314476\\
44.5025125628141	0.496420624707943\\
44.8643216080402	0.450649899872781\\
45.2261306532663	0.40308275234443\\
45.5879396984925	0.353908798726144\\
45.9497487437186	0.303324060821586\\
46.3115577889447	0.251530184234418\\
46.6733668341709	0.198733634549778\\
47.035175879397	0.145144874301894\\
47.3969849246231	0.0909775240086766\\
47.7587939698492	0.0364475106176874\\
48.1206030150754	-0.0182277932420295\\
48.4824120603015	-0.0728304357709972\\
48.8442211055276	-0.127142754819118\\
49.2060301507538	-0.180948245550855\\
49.5678391959799	-0.234032423497006\\
49.929648241206	-0.286183679552729\\
50.2914572864322	-0.337194123513488\\
50.6532663316583	-0.386860412786404\\
51.0150753768844	-0.434984562973476\\
51.3768844221105	-0.481374737095463\\
51.7386934673367	-0.525846010310361\\
52.1005025125628	-0.568221107078063\\
52.4623115577889	-0.608331107832669\\
52.8241206030151	-0.646016122345424\\
53.1859296482412	-0.68112592709407\\
53.5477386934673	-0.713520564097875\\
53.9095477386935	-0.743070898831193\\
54.2713567839196	-0.769659134991573\\
54.6331658291457	-0.793179284070365\\
54.9949748743719	-0.813537587854009\\
55.356783919598	-0.830652892171782\\
55.7185929648241	-0.844456970400119\\
56.0804020100502	-0.854894795433956\\
56.4422110552764	-0.861924759040907\\
56.8040201005025	-0.865518837723887\\
57.1658291457286	-0.865662704430988\\
57.5276381909548	-0.862355785667316\\
57.8894472361809	-0.855611263781103\\
58.251256281407	-0.845456024415005\\
58.6130653266332	-0.831930549332045\\
58.9748743718593	-0.815088755043424\\
59.3366834170854	-0.794997777881499\\
59.6984924623116	-0.771737706374665\\
60.0603015075377	-0.745401261990987\\
60.4221105527638	-0.716093429523235\\
60.78391959799	-0.683931038588694\\
61.1457286432161	-0.649042297912051\\
61.5075376884422	-0.6115662842478\\
61.8693467336683	-0.571652387979511\\
62.2311557788945	-0.529459717605932\\
62.5929648241206	-0.485156465487842\\
62.9547738693467	-0.438919237383971\\
63.3165829145729	-0.390932348448628\\
63.678391959799	-0.341387088497442\\
64.0402010050251	-0.290480959470036\\
64.4020100502513	-0.238416888129364\\
64.7638190954774	-0.185402417136113\\
65.1256281407035	-0.131648877722764\\
65.4874371859296	-0.0773705472653256\\
65.8492462311558	-0.0227837951108405\\
66.2110552763819	0.0318937799342869\\
66.572864321608	0.0864442170169684\\
66.9346733668342	0.140650062093333\\
67.2964824120603	0.194295234764324\\
67.6582914572864	0.247165889635561\\
68.0201005025126	0.299051268767807\\
68.3819095477387	0.349744541819938\\
68.7437185929648	0.399043630535356\\
69.1055276381909	0.446752014285192\\
69.4673366834171	0.492679513457187\\
69.8291457286432	0.536643047567405\\
70.1909547738694	0.578467365072749\\
70.5527638190955	0.617985741975025\\
70.9145728643216	0.655040646431696\\
71.2763819095477	0.689484366724024\\
71.6381909547739	0.721179600079305\\
72	0.75\\
};
\end{axis}
\end{tikzpicture}%}
\hfill
\setlength\fwidth{0.45\textwidth}
\subfloat{
% This file was created by matlab2tikz.
%
%The latest updates can be retrieved from
%  http://www.mathworks.com/matlabcentral/fileexchange/22022-matlab2tikz-matlab2tikz
%where you can also make suggestions and rate matlab2tikz.
%
\definecolor{mycolor1}{rgb}{1.00000,1.00000,0.00000}%
%
\begin{tikzpicture}

\begin{axis}[%
width=0.951\fwidth,
height=0.75\fwidth,
at={(0\fwidth,0\fwidth)},
scale only axis,
xmin=0,
xmax=36,
xtick={0,6,12,18,24,30},
xticklabels={{ },{1},{ },{2},{ },{3},{ }},
xlabel={Slot number},
xmajorgrids,
ymin=-1.75,
ymax=1,
ytick={-1.75,-1.5,-1.25,-1,-0.75,-0.5,-0.25,0,0.25,0.5,0.75,1},
yticklabels={{ },{ },{ },{-1.0},{ },{-0.5},{ },{ },{ },{0.5},{ },{1.0}},
ymajorgrids,
axis background/.style={fill=white}
]
\addplot [color=blue,solid,forget plot]
  table[row sep=crcr]{%
0	-1.75\\
3	-1.75\\
3	-1.25\\
9	-1.25\\
9	-1.75\\
12	-1.75\\
};

\addplot[area legend,solid,draw=black,fill=red,forget plot]
table[row sep=crcr] {%
x	y\\
6	-1.75\\
6	-1.25\\
9	-1.25\\
9	-1.75\\
6	-1.75\\
}--cycle;

\addplot[area legend,solid,draw=black,fill=blue,forget plot]
table[row sep=crcr] {%
x	y\\
3	-1.75\\
3	-1.25\\
6	-1.25\\
6	-1.75\\
3	-1.75\\
}--cycle;
\addplot [color=blue,solid,forget plot]
  table[row sep=crcr]{%
12	-1.75\\
15	-1.75\\
15	-1.25\\
21	-1.25\\
21	-1.75\\
24	-1.75\\
};

\addplot[area legend,solid,draw=black,fill=mycolor1,forget plot]
table[row sep=crcr] {%
x	y\\
18	-1.75\\
18	-1.25\\
21	-1.25\\
21	-1.75\\
18	-1.75\\
}--cycle;

\addplot[area legend,solid,draw=black,fill=red,forget plot]
table[row sep=crcr] {%
x	y\\
15	-1.75\\
15	-1.25\\
18	-1.25\\
18	-1.75\\
15	-1.75\\
}--cycle;
\addplot [color=blue,solid,forget plot]
  table[row sep=crcr]{%
24	-1.75\\
27	-1.75\\
27	-1.25\\
33	-1.25\\
33	-1.75\\
36	-1.75\\
};

\addplot[area legend,solid,draw=black,fill=blue,forget plot]
table[row sep=crcr] {%
x	y\\
30	-1.75\\
30	-1.25\\
33	-1.25\\
33	-1.75\\
30	-1.75\\
}--cycle;

\addplot[area legend,solid,draw=black,fill=mycolor1,forget plot]
table[row sep=crcr] {%
x	y\\
27	-1.75\\
27	-1.25\\
30	-1.25\\
30	-1.75\\
27	-1.75\\
}--cycle;

\addplot[area legend,solid,table/row sep=crcr,patch,patch type=rectangle,fill=black,faceted color=black,forget plot,patch table={%
0	1	2	3\\
4	5	6	7\\
8	9	10	11\\
}]
table[row sep=crcr] {%
x	y\\
3	0\\
3	0.75\\
9	0.75\\
9	0\\
15	0\\
15	-0.75\\
21	-0.75\\
21	0\\
27	0\\
27	3.33066907387547e-016\\
33	3.33066907387547e-016\\
33	0\\
};
\addplot [color=black,solid,forget plot]
  table[row sep=crcr]{%
0	0\\
36	0\\
};
\addplot [color=black,dashed,forget plot]
  table[row sep=crcr]{%
0	0.75\\
0.363636363636364	0.775953370168997\\
0.727272727272727	0.798782249984231\\
1.09090909090909	0.818394715603996\\
1.45454545454545	0.834711794551338\\
1.81818181818182	0.84766778370835\\
2.18181818181818	0.85721051387943\\
2.54545454545455	0.863301559858227\\
2.90909090909091	0.865916395152381\\
3.27272727272727	0.865044490743052\\
3.63636363636364	0.860689357481564\\
4	0.852868531952443\\
4.36363636363636	0.841613505859783\\
4.72727272727273	0.826969599221269\\
5.09090909090909	0.808995777880458\\
5.45454545454546	0.787764416072141\\
5.81818181818182	0.763361004996844\\
6.18181818181818	0.735883808577933\\
6.54545454545454	0.705443467787485\\
6.90909090909091	0.672162555134133\\
7.27272727272727	0.636175081106841\\
7.63636363636364	0.597625954561962\\
8	0.556670399226419\\
8.36363636363636	0.51347332866654\\
8.72727272727273	0.468208682239332\\
9.09090909090909	0.421058724700084\\
9.45454545454546	0.372213312286522\\
9.81818181818182	0.321869128234751\\
10.1818181818182	0.270228890805282\\
10.5454545454545	0.217500537008143\\
10.9090909090909	0.163896385313932\\
11.2727272727273	0.109632280722269\\
11.6363636363636	0.0549267256301677\\
12	-1.59081031806902e-016\\
12.3636363636364	-0.054926725630168\\
12.7272727272727	-0.109632280722269\\
13.0909090909091	-0.163896385313931\\
13.4545454545455	-0.217500537008143\\
13.8181818181818	-0.270228890805282\\
14.1818181818182	-0.321869128234751\\
14.5454545454545	-0.372213312286522\\
14.9090909090909	-0.421058724700084\\
15.2727272727273	-0.468208682239333\\
15.6363636363636	-0.51347332866654\\
16	-0.556670399226419\\
16.3636363636364	-0.597625954561962\\
16.7272727272727	-0.636175081106841\\
17.0909090909091	-0.672162555134133\\
17.4545454545455	-0.705443467787485\\
17.8181818181818	-0.735883808577934\\
18.1818181818182	-0.763361004996843\\
18.5454545454545	-0.787764416072141\\
18.9090909090909	-0.808995777880458\\
19.2727272727273	-0.826969599221269\\
19.6363636363636	-0.841613505859783\\
20	-0.852868531952443\\
20.3636363636364	-0.860689357481564\\
20.7272727272727	-0.865044490743052\\
21.0909090909091	-0.865916395152381\\
21.4545454545455	-0.863301559858227\\
21.8181818181818	-0.85721051387943\\
22.1818181818182	-0.84766778370835\\
22.5454545454545	-0.834711794551338\\
22.9090909090909	-0.818394715603996\\
23.2727272727273	-0.798782249984231\\
23.6363636363636	-0.775953370168997\\
24	-0.75\\
24.3636363636364	-0.721026644538829\\
24.7272727272727	-0.689149969261962\\
25.0909090909091	-0.654498330290064\\
25.4545454545455	-0.617211257543195\\
25.8181818181818	-0.577438892903068\\
26.1818181818182	-0.535341385644679\\
26.5454545454545	-0.491088247571705\\
26.9090909090909	-0.444857670452296\\
27.2727272727273	-0.396835808503719\\
27.6363636363636	-0.347216028815024\\
28	-0.296198132726024\\
28.3636363636364	-0.243987551297821\\
28.7272727272727	-0.190794518114429\\
29.0909090909091	-0.136833222746325\\
29.4545454545455	-0.0823209482846555\\
29.8181818181818	-0.0274771964189097\\
30.1818181818182	0.0274771964189098\\
30.5454545454545	0.0823209482846556\\
30.9090909090909	0.136833222746326\\
31.2727272727273	0.190794518114429\\
31.6363636363636	0.243987551297821\\
32	0.296198132726023\\
32.3636363636364	0.347216028815024\\
32.7272727272727	0.396835808503719\\
33.0909090909091	0.444857670452296\\
33.4545454545455	0.491088247571704\\
33.8181818181818	0.535341385644679\\
34.1818181818182	0.577438892903068\\
34.5454545454545	0.617211257543195\\
34.9090909090909	0.654498330290064\\
35.2727272727273	0.689149969261962\\
35.6363636363636	0.721026644538829\\
36	0.75\\
};
\end{axis}
\end{tikzpicture}%}
\\
\fontsize{8}{0}\selectfont
\setlength\fwidth{0.45\textwidth}
\subfloat{
% This file was created by matlab2tikz.
%
%The latest updates can be retrieved from
%  http://www.mathworks.com/matlabcentral/fileexchange/22022-matlab2tikz-matlab2tikz
%where you can also make suggestions and rate matlab2tikz.
%
\definecolor{mycolor1}{rgb}{1.00000,1.00000,0.00000}%
%
\begin{tikzpicture}

\begin{axis}[%
width=0.951\fwidth,
height=0.514\fwidth,
at={(0\fwidth,0\fwidth)},
scale only axis,
xmin=0,
xmax=72,
xtick={0,6,12,18,24,30,36,42,48,54,60,66},
xticklabels={{ },{1},{ },{2},{ },{3},{ },{4},{ },{5},{ },{6},{ }},
xlabel={Slot number},
xmajorgrids,
ymin=-1.75,
ymax=1,
ytick={-1.75,-1.5,-1.25,-1,-0.75,-0.5,-0.25,0,0.25,0.5,0.75,1},
yticklabels={{ },{ },{ },{-1.0},{ },{-0.5},{ },{ },{ },{0.5},{ },{1.0}},
ymajorgrids,
axis background/.style={fill=white}
]
\addplot [color=blue,solid,forget plot]
  table[row sep=crcr]{%
0	-1.75\\
3	-1.75\\
3	-1.25\\
9	-1.25\\
9	-1.75\\
12	-1.75\\
};

\addplot[area legend,solid,draw=black,fill=red,forget plot]
table[row sep=crcr] {%
x	y\\
3	-1.75\\
3	-1.25\\
9	-1.25\\
9	-1.75\\
3	-1.75\\
}--cycle;
\addplot [color=blue,solid,forget plot]
  table[row sep=crcr]{%
12	-1.75\\
15	-1.75\\
15	-1.25\\
21	-1.25\\
21	-1.75\\
24	-1.75\\
};

\addplot[area legend,solid,draw=black,fill=blue,forget plot]
table[row sep=crcr] {%
x	y\\
15	-1.75\\
15	-1.25\\
21	-1.25\\
21	-1.75\\
15	-1.75\\
}--cycle;
\addplot [color=blue,solid,forget plot]
  table[row sep=crcr]{%
24	-1.75\\
27	-1.75\\
27	-1.25\\
33	-1.25\\
33	-1.75\\
36	-1.75\\
};

\addplot[area legend,solid,draw=black,fill=mycolor1,forget plot]
table[row sep=crcr] {%
x	y\\
27	-1.75\\
27	-1.25\\
33	-1.25\\
33	-1.75\\
27	-1.75\\
}--cycle;
\addplot [color=blue,solid,forget plot]
  table[row sep=crcr]{%
36	-1.75\\
39	-1.75\\
39	-1.25\\
45	-1.25\\
45	-1.75\\
48	-1.75\\
};

\addplot[area legend,solid,draw=black,fill=red,forget plot]
table[row sep=crcr] {%
x	y\\
39	-1.75\\
39	-1.25\\
45	-1.25\\
45	-1.75\\
39	-1.75\\
}--cycle;
\addplot [color=blue,solid,forget plot]
  table[row sep=crcr]{%
48	-1.75\\
51	-1.75\\
51	-1.25\\
57	-1.25\\
57	-1.75\\
60	-1.75\\
};

\addplot[area legend,solid,draw=black,fill=blue,forget plot]
table[row sep=crcr] {%
x	y\\
51	-1.75\\
51	-1.25\\
57	-1.25\\
57	-1.75\\
51	-1.75\\
}--cycle;
\addplot [color=blue,solid,forget plot]
  table[row sep=crcr]{%
60	-1.75\\
63	-1.75\\
63	-1.25\\
69	-1.25\\
69	-1.75\\
72	-1.75\\
};

\addplot[area legend,solid,draw=black,fill=mycolor1,forget plot]
table[row sep=crcr] {%
x	y\\
63	-1.75\\
63	-1.25\\
69	-1.25\\
69	-1.75\\
63	-1.75\\
}--cycle;

\addplot[area legend,solid,table/row sep=crcr,patch,patch type=rectangle,fill=black,faceted color=black,forget plot,patch table={%
0	1	2	3\\
4	5	6	7\\
8	9	10	11\\
12	13	14	15\\
16	17	18	19\\
20	21	22	23\\
}]
table[row sep=crcr] {%
x	y\\
3	0\\
3	1\\
9	1\\
9	0\\
15	0\\
15	0.5\\
21	0.5\\
21	0\\
27	0\\
27	-0.5\\
33	-0.5\\
33	0\\
39	0\\
39	-1\\
45	-1\\
45	0\\
51	0\\
51	-0.5\\
57	-0.5\\
57	0\\
63	0\\
63	0.5\\
69	0.5\\
69	0\\
};
\addplot [color=black,solid,forget plot]
  table[row sep=crcr]
\hfill
\fontsize{8}{0}\selectfont
\setlength\fwidth{0.45\textwidth}
\subfloat{
% This file was created by matlab2tikz.
%
%The latest updates can be retrieved from
%  http://www.mathworks.com/matlabcentral/fileexchange/22022-matlab2tikz-matlab2tikz
%where you can also make suggestions and rate matlab2tikz.
%
\definecolor{mycolor1}{rgb}{1.00000,1.00000,0.00000}%
%
\begin{tikzpicture}

\begin{axis}[%
width=0.951\fwidth,
height=0.514\fwidth,
at={(0\fwidth,0\fwidth)},
scale only axis,
xmin=0,
xmax=72,
xtick={0,6,12,18,24,30,36,42,48,54,60,66},
xticklabels={{ },{1},{ },{2},{ },{3},{ },{4},{ },{5},{ },{6},{ }},
xlabel={Slot number},
xmajorgrids,
ymin=-1.75,
ymax=1,
ytick={-1.75,-1.5,-1.25,-1,-0.75,-0.5,-0.25,0,0.25,0.5,0.75,1},
yticklabels={{ },{ },{ },{-1.0},{ },{-0.5},{ },{ },{ },{0.5},{ },{1.0}},
ymajorgrids,
axis background/.style={fill=white}
]
\addplot [color=blue,solid,forget plot]
  table[row sep=crcr]{%
0	-1.75\\
3	-1.75\\
3	-1.25\\
9	-1.25\\
9	-1.75\\
12	-1.75\\
};

\addplot[area legend,solid,draw=black,fill=red,forget plot]
table[row sep=crcr] {%
x	y\\
3	-1.5\\
3	-1.25\\
9	-1.25\\
9	-1.5\\
3	-1.5\\
}--cycle;

\addplot[area legend,solid,draw=black,fill=red,forget plot]
table[row sep=crcr] {%
x	y\\
3	-1.75\\
3	-1.5\\
9	-1.5\\
9	-1.75\\
3	-1.75\\
}--cycle;
\addplot [color=blue,solid,forget plot]
  table[row sep=crcr]{%
12	-1.75\\
15	-1.75\\
15	-1.25\\
21	-1.25\\
21	-1.75\\
24	-1.75\\
};

\addplot[area legend,solid,draw=black,fill=blue,forget plot]
table[row sep=crcr] {%
x	y\\
15	-1.5\\
15	-1.25\\
21	-1.25\\
21	-1.5\\
15	-1.5\\
}--cycle;

\addplot[area legend,solid,draw=black,fill=blue,forget plot]
table[row sep=crcr] {%
x	y\\
15	-1.75\\
15	-1.5\\
21	-1.5\\
21	-1.75\\
15	-1.75\\
}--cycle;
\addplot [color=blue,solid,forget plot]
  table[row sep=crcr]{%
24	-1.75\\
27	-1.75\\
27	-1.25\\
33	-1.25\\
33	-1.75\\
36	-1.75\\
};

\addplot[area legend,solid,draw=black,fill=mycolor1,forget plot]
table[row sep=crcr] {%
x	y\\
27	-1.5\\
27	-1.25\\
33	-1.25\\
33	-1.5\\
27	-1.5\\
}--cycle;

\addplot[area legend,solid,draw=black,fill=mycolor1,forget plot]
table[row sep=crcr] {%
x	y\\
27	-1.75\\
27	-1.5\\
33	-1.5\\
33	-1.75\\
27	-1.75\\
}--cycle;
\addplot [color=blue,solid,forget plot]
  table[row sep=crcr]{%
36	-1.75\\
39	-1.75\\
39	-1.25\\
45	-1.25\\
45	-1.75\\
48	-1.75\\
};

\addplot[area legend,solid,draw=black,fill=red,forget plot]
table[row sep=crcr] {%
x	y\\
39	-1.5\\
39	-1.25\\
45	-1.25\\
45	-1.5\\
39	-1.5\\
}--cycle;

\addplot[area legend,solid,draw=black,fill=red,forget plot]
table[row sep=crcr] {%
x	y\\
39	-1.75\\
39	-1.5\\
45	-1.5\\
45	-1.75\\
39	-1.75\\
}--cycle;
\addplot [color=blue,solid,forget plot]
  table[row sep=crcr]{%
48	-1.75\\
51	-1.75\\
51	-1.25\\
57	-1.25\\
57	-1.75\\
60	-1.75\\
};

\addplot[area legend,solid,draw=black,fill=blue,forget plot]
table[row sep=crcr] {%
x	y\\
51	-1.5\\
51	-1.25\\
57	-1.25\\
57	-1.5\\
51	-1.5\\
}--cycle;

\addplot[area legend,solid,draw=black,fill=blue,forget plot]
table[row sep=crcr] {%
x	y\\
51	-1.75\\
51	-1.5\\
57	-1.5\\
57	-1.75\\
51	-1.75\\
}--cycle;
\addplot [color=blue,solid,forget plot]
  table[row sep=crcr]{%
60	-1.75\\
63	-1.75\\
63	-1.25\\
69	-1.25\\
69	-1.75\\
72	-1.75\\
};

\addplot[area legend,solid,draw=black,fill=mycolor1,forget plot]
table[row sep=crcr] {%
x	y\\
63	-1.5\\
63	-1.25\\
69	-1.25\\
69	-1.5\\
63	-1.5\\
}--cycle;

\addplot[area legend,solid,draw=black,fill=mycolor1,forget plot]
table[row sep=crcr] {%
x	y\\
63	-1.75\\
63	-1.5\\
69	-1.5\\
69	-1.75\\
63	-1.75\\
}--cycle;

\addplot[area legend,solid,table/row sep=crcr,patch,patch type=rectangle,fill=black,faceted color=black,forget plot,patch table={%
0	1	2	3\\
4	5	6	7\\
8	9	10	11\\
12	13	14	15\\
16	17	18	19\\
20	21	22	23\\
}]
table[row sep=crcr] {%
x	y\\
3	0\\
3	1\\
9	1\\
9	0\\
15	0\\
15	0.5\\
21	0.5\\
21	0\\
27	0\\
27	-0.5\\
33	-0.5\\
33	0\\
39	0\\
39	-1\\
45	-1\\
45	0\\
51	0\\
51	-0.5\\
57	-0.5\\
57	0\\
63	0\\
63	0.5\\
69	0.5\\
69	0\\
};
\addplot [color=black,solid,forget plot]
  table[row sep=crcr]
\caption{Test results using \texttt{arun.m}}
\label{fig:tests}
\end{figure}

\end{appendices}