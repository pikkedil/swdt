\chapter{Revision of the Air Gap Element}\label{app:AGE}
The use of an air gap element (AGE) in the finite element analysis means that the air gap mesh is 
replaced. It is also very accurate since it is derived from the analytical solution, and since there 
is no mesh in the air gap there are no limits on rotor angular movement. In this section the graphical 
presentation and the AGE matrix are reviewed. 

\section{Introductory remarks}
\label{sec:introduction}
In the design of electrical machines a detailed analysis is required of the magnetic field 
distribution to predict the machine's performance. This then generally leads to the calculation 
of electromagnetic forces and torque. The problem when solving the magnetic field using the finite 
element method is the rotation of the rotor and its influence on the mesh after each movement. There 
are several ways to achieve movement of the rotor past the stator. Some of them include the boundary 
element technique, coupled elements, slip surface, pre-stored meshes and a moving-band.

\cite{razek_1981} introduced an air gap element (AGE) for the finite element analysis of rotating 
electrical machines. It allows to keep a constant topology while the rotor is moved. The AGE replaces 
the mesh in the air gap and is based on the analytical solution of the field in the air gap combined 
with a finite element solution of the field in the non-linear parts of the stator and rotor. 
\cite{felianchi_1983} extended the AGE to be consistent with second order iso-parametric finite 
elements.

Advantages of the method are that the model of the air gap is almost perfect and the accuracy depends 
only on the quality of the discretisation of the stator and rotor. However, due to the computing time 
necessary to compute the AGE matrix the total calculation time is longer compared to that of the 
classical finite element method. \cite{flack_1994} introduced a time-saving computational scheme which 
enables a rapid re-calculation of the AGE stiffness terms and thereby a reduced time to obtain a field 
solution. The idea of the AGE was initially derived for the polar coordinate system and 
\cite{wang_2002} derived the AGE for cartesian planes. 

From most recent publications by \cite{kalokirsis_2005} it can be summarised that the AGE is still 
in use. \cite{de_gersem_2005} improved the computational time to be as efficient as the moving-band 
and sliding-surface methods.

\section{Graphical presentation of the AGE}\label{sec:AGE_element}
A finite element mesh that includes the air gap element is shown in Fig.~\ref{fig:f_mesh}. The AGE 
is uniform and the inner and outer radius falls on $r = a$ and $r = b$ respectively. Both the stator 
and rotor are meshed and the AGE is the element bounded by the nodes lying on both sides of the air 
gap. A solution domain corresponding to positive periodicity of the vector potential is used to derive 
the air gap element theory and is defined by $\theta_0$.

The side lying on the negative $x$-axis is referred to as the master\footnote{The FE program assumes 
the master boundary to lie on the negative $x$-axis.} and the side lying on $\pi-\theta_0$ is the 
slave. The nodes on the rotor outer radius are numbered from $1$ to $s$, and the nodes on the stator 
inner radius from $s+1$ to $t$. Due to periodicity, the nodes of the AGE lying in the slave is not 
used. The nodes are saved in polar coordinates. The positive rotation direction is counter-clockwise.

\begin{figure}
	\centering
		\input{figs/AppC/f_mesh.tex}
	\caption{Finite element mesh with Air Gap Element}	
	\label{fig:f_mesh}  
\end{figure}

%-------------------------------
\subsection{AGE stiffness matrix}
\label{AGE_stiffness}
The general form for the AGE stiffness terms are given by \cite{razek_1982} as
\begin{equation}\label{eqn:stiffness}
    S_{ij}=\frac{\theta_{0}}{4}\frac
    {ln\left(\frac{b}{c'}\right) - ln\left(\frac{a}{c'}\right)}
    {ln\left(\frac{c}{c'}\right)   ln\left(\frac{e}{e'}\right)}a_{oi}a_{oj} +
    \frac{\theta_{0}}{2}\sum^{\infty}_{n=1}f(\lambda_{n})
    \left(a_{ni}a_{nj}+b_{ni}b_{nj}\right)
\end{equation}
given that
\begin{equation}\label{eqn:func_aoi_lambda}
    a_{oi}=\frac{\theta_{i+1}-\theta_{i-1}}{\theta_{0}}, \; 
    \lambda_{n}=\frac{2\pi n}{\theta_{0}}
    \; \textnormal{and} \;
    f(\lambda_{n})=\frac{\lambda_{n}}{F_{ce}}\left(F_{bce}-F_{ace}\right)
\end{equation}
where
\begin{equation}\label{eqn:fce}
    F_{ce}=
    \left[ 
        \left(\frac{c}{c'}\right)^{\lambda_{n}}-
        \left(\frac{c'}{c}\right)^{\lambda_{n}}  
    \right]
    \left[ 
        \left(\frac{e}{e'}\right)^{\lambda_{n}}-
        \left(\frac{e'}{e}\right)^{\lambda_{n}}
    \right] \nonumber
\end{equation}
\begin{equation}
\label{eqn:fbce}
	F_{bce}= 
	  \left[ 
	    \left(\frac{b}{c'}\right)^{\lambda_{n}}-
	    \left(\frac{c'}{b}\right)^{\lambda_{n}}
	  \right]  
	  \left[
	    \left(\frac{b}{e'}\right)^{\lambda_{n}}+
	    \left(\frac{e'}{b}\right)^{\lambda_{n}}
	  \right]	\nonumber   
\end{equation}
\begin{equation}
\label{eqn:face}
	F_{ace}=
	   \left[
	     \left(\frac{a}{c'}\right)^{\lambda_{n}}-
	     \left(\frac{c'}{a}\right)^{\lambda_{n}}
	   \right]
	   \left[
	     \left(\frac{a}{e'}\right)^{\lambda_{n}}+
	     \left(\frac{e'}{a}\right)^{\lambda_{n}}
	   \right] \nonumber
\end{equation}
The Fourier coefficients in \eqref{eqn:stiffness} are calculated as follows
\begin{equation}\label{eqn:ani}
    a_{ni}=-k_{n}
    \left[
        \frac{1}{\phi_{1}}
        \sin\left(\frac{\lambda_{n}}{2}\phi_{2}\right) 
        \sin\left(\frac{\lambda_{n}}{2}\phi_{1}\right)+ 
        \frac{1}{\phi_{3}}
        \sin\left(\frac{\lambda_{n}}{2}\phi_{4}\right)
        \sin\left(\frac{-\lambda_{n}}{2}\phi_{3}\right)
    \right]
\end{equation}
\begin{equation}\label{eqn:bni}
    b_{ni}=k_{n}
    \left[
        \frac{1}{\phi_{1}}
        \sin\left(\frac{\lambda_{n}}{2}\phi_{1}\right) 
        \cos\left(\frac{\lambda_{n}}{2}\phi_{2}\right) + 
        \frac{1}{\phi_{3}}
        \sin\left(\frac{-\lambda_{n}}{2}\phi_{3}\right)
        \cos\left(\frac{\lambda_{n}}{2}\phi_{4}\right)
    \right]
\end{equation}
where
\begin{eqnarray}\label{eqn:theta}
  \phi_{1} = \theta_{i}-\theta_{i-1};        &  
  \phi_{2} = \theta_{i}+\theta_{i-1}         \nonumber \\  
  \phi_{3} = \theta_{i}-\theta_{i+1};        &  
  \phi_{4} = \theta_{i}+\theta_{i+1}         \nonumber \\
  k_{n}    = \frac{4}{\lambda^{2}_{n}\theta_{0}} \nonumber & \;
\end{eqnarray}

%-----------------------------
\subsection{Time-saving scheme}
%-----------------------------
\label{subsec:time_saving}
The computational time can be reduced if the terms in the AGE matrix that are independent of 
rotor angular movement are stored. Only the AGE matrix terms dependent on rotor angular movement 
need to be recalculated for each $\Delta\theta$. This, however, implies more storage  but the 
recalculation time is remarkable as shown by \cite{flack_1994}. 

From the Fourier coefficients given in \eqref{eqn:ani} and \eqref{eqn:bni} 
$\phi_1$ and $\phi_3$ are unchanged when the rotor is moved through an
angle $\Delta\theta$, whereas $\phi_2$ and $\phi_4$ change as follows: 
$\phi_2\rightarrow\phi_2+2\Delta\theta$ and 
$\phi_4\rightarrow\phi_4+2\Delta\theta$. For each of the Fourier coefficients a time-saving scheme will be introduced.

\subsubsection{Simplified Fourier coefficients}
Using $\sin(-x)=-\sin(x)$, \eqref{eqn:ani} can be rewritten as 
\begin{equation}\label{eqn:ani_new}
    a_{ni}= C_{ni}\sin\left(\frac{\lambda_{n}}{2}\phi_{2}\right)+
    D_{ni}\sin\left(\frac{\lambda_{n}}{2}\phi_{4}\right)
    \qquad
    \begin{cases}
        \vspace{0.5em}
        C_{ni}=\frac{k_{n}}{\phi_{1}}\sin\left(\frac{-\lambda_{n}}{2} \phi_{1}\right) \\
        D_{ni}=\frac{k_{n}}{\phi_{3}}\sin\left(\frac{\lambda_{n}}{2} \phi_{3}\right)  \\
    \end{cases}                    
\end{equation}
Introducing movement \eqref{eqn:ani_new} changes as follows
\begin{equation}\label{eqn:ani_move}
    a_{ni|_{+\Delta\theta}}=
    C_{ni}\sin\left(
    \frac{\lambda_{n}}{2}\left(\phi_{2}+2\Delta\theta\right)\right)+
    D_{ni}\sin\left(
    \frac{\lambda_{n}}{2}\left(\phi_{4}+2\Delta\theta\right)\right)
\end{equation}
and can be rewritten when simplified using trigonometric identities\footnote
{
    $\sin(x\pm y)=\sin(x)\cos(y)\pm \cos(x)\sin(y)$ and
    $\cos(x\pm y)=\cos(x)\cos(y)\mp \sin(x)\sin(y)$
} 
as
\begin{equation}\label{eqn:ani_short}
    a_{ni|_{+\Delta\theta}}=G_{ni}\cos\left(\lambda_{n}\Delta\theta\right)+
        F_{ni}\sin\left(\lambda_{n}\Delta\theta\right)
    \qquad
    \begin{cases}
        \vspace{0.5em}
        F_{ni}=C_{ni}\cos\left(\frac{\lambda_{n}}{2}\theta_{1}\right)+
            D_{ni}\cos\left(\frac{\lambda_{n}}{2}\theta_{3}\right)
        G_{ni}=C_{ni}\sin\left(\frac{\lambda_{n}}{2}\theta_{1}\right)+
            D_{ni}\sin\left(\frac{\lambda_{n}}{2}\theta_{3}\right)
    \end{cases}
\end{equation}
Equation \eqref{eqn:ani_short} can be written in a compact form using a trigonometric identity\footnote
{
    $A\cos(x)\pm B\sin(x)=R\cos\left(x+\theta\right)$ where
    $R=\sqrt{A^{2}+B^{2}}\) and \(\theta=\tan^{-1}\left(\mp \frac{B}{A}\right)$
}
as
\begin{equation}\label{eqn:ani_save}
    a_{ni|_{+\Delta\theta}}=H_{ni}\cos
    \left(\lambda_{n}\Delta\theta+P_{ni}\right)
    \qquad
    \begin{cases}
        \vspace{0.5em}
        H_{ni}=\sqrt{F^{2}_{ni}+G^{2}_{ni}} \\
        P_{ni} = \frac{\lambda_{n}}{2}\theta_{2}-\tan^{-1}\frac{F_{ni}}{G_{ni}}
    \end{cases}
\end{equation} 

A similar method that was used to simplify $a_{ni}$ will now be introduced to the coefficient $b_{ni}$. 
Simplifying \eqref{eqn:bni} results in
\begin{equation}\label{eqn:bni_new}
    b_{ni}=C_{ni}\cos\left(\frac{\lambda_{n}}{2}\phi_{2}\right)+
    D_{ni}\cos\left(\frac{\lambda_{n}}{2}\phi_{4}\right)
    \qquad
    \begin{cases}
        \vspace{0.5em}
        C_{ni}=\frac{k_{n}}{\phi_{1}}\sin\left(\frac{\lambda_{n}}{2} \phi_{1}\right) \\ 
        D_{ni}=\frac{k_{n}}{\phi_{3}}\sin\left(\frac{-\lambda_{n}}{2} \phi_{3}\right)
    \end{cases}
\end{equation} 
If rotor movement is introduced the coefficient will change as follows
\begin{equation}\label{eqn:bni_move}
    b_{ni|_{+\Delta\theta}}=
        C_{ni}\cos\left(
        \frac{\lambda_{n}}{2}\left(\phi_{2}+2\Delta\theta\right)\right)+
        D_{ni}\cos\left(
        \frac{\lambda_{n}}{2}\left(\phi_{4}+2\Delta\theta\right)\right)
\end{equation}  
and using a trigonometric identity it simplifies to
\begin{equation}\label{eqn:bni_short}
    b_{ni|_{+\Delta\theta}}=F_{ni}\cos\left(\lambda_{n}\Delta\theta\right)-
    G_{ni}\sin\left(\lambda_{n}\Delta\theta\right)
\end{equation}
Equation \eqref{eqn:bni_short} can further be simplified to minimise the computational effort as 
was done in \eqref{eqn:ani_save}. This results in
\begin{equation}\label{eqn:bni_save}
    b_{ni|_{+\Delta\theta}}=H_{ni}\cos
    \left(\lambda_{n}\Delta\theta+P_{ni}\right)
    \qquad
    \begin{cases}
        \vspace{0.5em}
        H_{ni}=\sqrt{F^{2}_{ni}+G^{2}_{ni}} \; \textnormal{and} \\
        P_{ni} = \frac{\lambda_{n}}{2}\theta_{2}-\tan^{-1}\frac{G_{ni}}{F_{ni}}
    \end{cases}
\end{equation}
The constants $\lambda_{n}$, $H_{ni}$ and $P_{ni}$ need to be pre-determined. It requires more 
storage but in case of movement $\Delta\theta$ the Fourier coefficients can be re-calculated in 
a much shorter time which significantly reduces the total calculation time. 

\subsubsection{Stepped AGE}
\label{subsubsec:stepped_age}
An implementation of the time-saving scheme implies that the constants given in  
\eqref{eqn:ani_save} and \eqref{eqn:bni_save} need to be stored in memory. This only holds for 
the nodes that are stationary. In the program the stator is chosen to rotate with counter-clockwise 
taken as positive\footnote{This implies that the relative rotation of the rotor with respect to the 
stator is clockwise.}. Thus the Fourier coefficients of the rotor nodes $(1-s)$ are saved. For a 
movement $\Delta\theta$ the new Fourier coefficients for the stator AGE nodes are re-calculated 
using \eqref{eqn:ani_save} and \eqref{eqn:bni_save} for $a_{ni}$ and $b_{ni}$ respectively. This 
holds for nodes $(s+1)$ to $t$.

\subsection{Periodicity conditions}\label{subsec:periodicity}
In some problems the symmetry can be used to reduce the size of the model. When possible only one 
pole-pitch is modeled which means that the periodicity condition must be negative. 
Fig.~\ref{fig:f_vec_2} shows a simple example of a field solution having positive periodicity. Using 
symmetry only half the model can be used. In such a case the $a_{oi}$ equals zero as shown by 
\cite{flack_1994} and \eqref{eqn:stiffness} reduces to 
\begin{equation} \label{eqn:stiffness_neg}
    S_{ij}=
    \frac{\theta_{0}}{2}\sum^{\infty}_{n=1}f(\lambda_{n})
    \left(a_{ni}a_{nj}+b_{ni}b_{nj}\right) \qquad \textnormal{for } n=1,3,5,\ldots
\end{equation}
where the Fourier coefficients have twice its value as given in \eqref{eqn:ani} and \eqref{eqn:bni}. 
If the equation for calculating the coefficients are kept the same in the case of negative periodicity 
the term containing the coefficients in \eqref{eqn:stiffness_neg} need to change as follows 
$(a_{ni}a_{nj}+b_{ni}b_{nj}) \rightarrow (2a_{ni}2a_{nj}+2b_{ni}2b_{nj})$ and 
\eqref{eqn:stiffness_neg} has the form
\begin{equation} \label{eqn:stiffness_neg_re}
    S_{ij}=
    4\frac{\theta_{0}}{2}\sum^{\infty}_{n=1}f(\lambda_{n})
    \left(a_{ni}a_{nj}+b_{ni}b_{nj}\right) \qquad \textnormal{for } n=1,3,5,\ldots
\end{equation}  

\begin{figure}
    \centering
    \input{figs/AppC/f_vec_2.tex}
    \caption{Vector potential solution with positive periodic boundary conditions}
    \label{fig:f_vec_2}
\end{figure}


\subsection{Electromagnetic torque}\label{torque}
The electromagnetic torque in an electrical machine is given by the Maxwell tensor as a function of 
the radial and tangential components of the flux density. The flux density in the air gap is calculated 
by derivatives of the expansion of the shape functions which then results in the quadratic product
\begin{equation}
\label{eqn:torque}
  T = -\frac{p\:l_{Fe}}{\mu_{0}}\mathbf{A}^{T}\mathbf{T}\mathbf{A}
\end{equation}
For simplicity the variables as presented by \cite{razek_1981} were changed as follows $c=c$, $c'=c$, 
$g=e'$, $g'=e$, and $k=\lambda_{n}$. The matrix \(\mathbf{T}\) is then given by 
\begin{equation}\label{eqn:tij_pos}
    t_{ij}=\frac{\theta_{0}}{2}\sum^{\infty}_{n=1}\lambda^{2}_{n}
    F_{cer}\left(a_{ni}b_{nj}-b_{ni}a_{nj}\right)
\end{equation}
and
\begin{equation}\label{eqn:fcer}
    F_{cer}=\frac
    {\left[
        \left(\frac{r}{c'}\right)^{\lambda_{n}}+
        \left(\frac{c'}{r}\right)^{\lambda_{n}}
    \right]
    \left[
        \left(\frac{r}{e'}\right)^{\lambda_{n}}-
        \left(\frac{e'}{r}\right)^{\lambda_{n}}
    \right]
  }
  { \left[
        \left(\frac{c}{c'}\right)^{\lambda_{n}}-
        \left(\frac{c'}{c}\right)^{\lambda_{n}}
    \right]
    \left[
        \left(\frac{e}{e'}\right)^{\lambda_{n}}-
        \left(\frac{e'}{e}\right)^{\lambda_{n}}
    \right]
  } 
\end{equation}  
In the case of negative periodicity the Fourier coefficients holds only for $n$ odd. The Fourier 
coefficients have twice its value compared to positive periodicity\footnote{If the calculating of 
the Fourier coefficients is done in the same way for both negative and positive periodicity it might 
be necessary to multiply \eqref{eqn:tij_neg} by a factor 4. This will however depend on how the factor 
2 for the Fourier coefficients in the case of negative periodicity is handled.}. For similar 
conditions as in \eqref{eqn:stiffness_neg_re}, \eqref{eqn:tij_pos} for negative periodicity changes 
to
\begin{equation}
  \label{eqn:tij_neg}
  t_{ij}=4\frac{\theta_{0}}{2}\sum^{\infty}_{n=1}\lambda^{2}_{n}
   F_{cer}\left(a_{ni}b_{nj}-b_{ni}a_{nj}\right) \qquad \textnormal{for } n=1,3,5,\ldots
\end{equation}
The torque ripple for the finite element mesh in Fig.~\ref{fig:f_mesh} is given in 
Fig.~\ref{fig:f_age_ss}. For the same condition as in the Cambridge software, the torque ripple was 
calculated with Maxwell 2D. 
\begin{figure}
    \centering
    \input{figs/AppC/f_age_ss.tex}
    \caption{Torque ripple calculation using the AGE}
    \label{fig:f_age_ss}
\end{figure}


\subsection{Air gap flux density}\label{subsec:air_gap_flux_density}
The air gap flux density can be calculated from the vector potential solution. In case of the AGE the 
nodes on $r=b$, the stator inner diameter nodes, are used. Nodes $(s+1)-t$ are already available from 
the AGE. For the complete solution the node $\left\langle r_{b},(\pi-\theta_{0})\right\rangle$ must 
be included. $B_{\delta}$ is then calculated as follows
\begin{equation}\label{eqn:bdelta}
    B_{\delta}=\frac{1}{r}
    \left(
        \frac{A_{z}(i+1)-A_{z}(i)}{\theta_{i+1}-\theta_{i}}
    \right)  
\end{equation}
Since the flux density as calculated in \eqref{eqn:bdelta} is obtained from differentiation, the 
result can be inaccurate for numerical reasons. To obtain better results it is preferable to perform 
a spatial harmonic analysis on the vector potential as presented in 
\ref{subsec:spatial_harmonic_analysis}. Differentiation of the harmonic components give better results.

\section{History of the Cambridge software}\label{sec:cambridge}
A short history of the development of the software at the University of Stellenbosch since 1993 will 
now be given. Among the post-graduate students in the Electrical Machines Laboratory the software is 
known as the Cambridge software. There is, however, no collaboration between the University of 
Stellenbosch and University of Cambridge in this regard. Since 1993 the software has been further 
developed individually.  

Albertus Volschenk, a South African citizen, did his Ph.D at Cambridge University in England. During 
his studies in the Department of Engineering, under the supervision of Prof.~Williamson, he used the 
finite element software to analyse a salient pole generator feeding a rectifier load. 
\cite{Volschenk1993} improved the calculation time of the air gap element. At this stage the software 
was written in Fortran running on a Unix machine. \cite{Williamson1995} presented the outcome of the 
research in a journal paper.

Upon his return to South Africa he left a copy of the source code in the Department of Electrical 
Engineering at Stellenbosch University. \cite{Kamper1996} adapted the software, still running under 
Unix, to perform an optimisation of a cage less synchronous reluctance machine. 

When the development of the software started, memory was still expensive and the data management was 
mainly done by means of the hard disk. \cite{Germishuizen2000} ported the software to a stand-alone 
PC and used the software to compare an induction machine with a synchronous reluctance machine for 
rail traction.

At present the software is capable of running on Window machines. The graphical output is viewed by 
software written in Matlab. 

\endinput