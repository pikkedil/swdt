\begin{abstract}%====================================================
At present most of all existing variable speed drive systems are still based on induction machines. In order to reduce energy loss, investment and maintenance costs there is a trend to replace the induction motor and gearbox with a direct drive system.
However, it was only until recently that the progress in material science made it possible to economically implement machines with permanent magnet technology in a direct drive system. Interior permanent magnets motors with single layer non-overlapping windings have advantages, which make it attractive for the use in direct drives. In the present dissertation a fast and accurate analysis methodology, suitable in an everyday design environment is offered. The method makes use of both the analytical and finite element method. Part of the design procedure requires a systematic algorithm to allocate the stator slots to the winding phase belts.
	
The accuracy of the proposed method was verified by means of a prototype traction machine. From the results it is concluded that three two-dimensional functions can be used to accurately calculate the performance of permanent magnet machines in an everyday design environment. The analysis method which was developed is based on real physical two-dimensional machine models. The unique contributions of the present dissertation are the performance calculation method and an expression that defines a winding phase belt.
\end{abstract}

\begin{abstract}[afrikaans]
Die meeste veranderlike spoed aandryfstelsels is tans gebaseer op induksiemasjiene. Om energie verliese, beleggings en onderhoudskoste te verminder, is daar 'n tendens om die induksiemasjien en ratkas te vervang met �n direkte aandryfstelsel. Dit is eers onlangs dat vooruitgang in die materiaal wetenskappe dit moontlik gemaak het om masjiene met permanent magneet tegnologie ekonomies in �n direkte aandryfstelsel te implementeer. Elektiese motors met permanente binnemagnete en nie-oorvleuelende wikkelings het voordele wat dit baie aantreklik maak vir die gebruik in direkte aandryfstelsels. In hierdie proefskrif word �n vinnige en akkurate analise metode geskik vir �n daaglikse ontwerpsomgewing voorgestel. Die metode maak gebruik van beide die analitise en die eindige element metode. �n Deel van die prosedure benodig �n sistematiese algoritme om die stator gleuwe aan die wikkeling-fase-gordels toe te ken.  
	
Die akkuraatheid van die metode word geverifieer deur middel van �n prototipe traksie motor. Uit die resultate word die gevolgtrekking gemaak dat drie twee-dimensionele funksies gebruik kan word om in �n daaglikse ontwerpsomgewing die werkverrigting van permanent magneet elektriese masjiene akkuraat te kan bereken. Die analisemetode wat ontwikkel is, is gebasseer op die werklike fisiese twee-dimensionele masjien model. Die unieke bydrae van hierdie proefskrif is die werkverrigtingsberekengsmetode en 'n uitdrukking wat 'n wikkeling-fase-gordel definieer.
\end{abstract}

\chapter{Acknowledgements}%==========================================

I would like to thank my promoters, Prof.~Maarten Kamper and Dr.~Andreas J�ckel, for their constant support and guidance throughout this project. I also acknowledge the following persons and foundation for their contributions:
\begin{itemize}
\item The external examiners Professors Bernd Ponick (University of Hanover, Germany),
      Konrad Reichert (Swiss Federal Institute of Technology Z�rich, Switzerland)~%
      and Frikkie van der Merwe (Stellenbosch University, South Africa) for~%
      their valuable comments and advice;
\item Prof.~Heinrich van der Mescht for continuous support with the dissertation~%
      layout;
\item My colleagues at Siemens AG (Vogelweiherstra�e, N�rnberg), particularly~%
      Peter Eckert and Thomas Schmidt for their assistance;
\item Prof.~Hans Otto Seinsch (University of Hanover) and Prof.~Zden\u{e}k~%
      \v{C}e\v{r}ovsk\'y (Technical University of Prague) who kindly provided me~%
      with details on Vil\'em Kl\'ima;
\item Ivan Kl\'ima for proof-reading my essay on his father, Vil\'em Kl\'ima, and
      for providing the photos of his father;
\item Willie Coetzee for his unceasing interest in my work and
\item The Harry Crossley Foundation for a valuable scholarship.      
\end{itemize}

\chapter{Dedications}%===============================================
 \vfill
 \begin{Afr}
 \begin{center}\itshape
    This dissertation is dedicated to Gielie, Joey and Paul 
 \end{center}
 \end{Afr}
 \vfill
 \clearpage

%====================================================================
\endinput
